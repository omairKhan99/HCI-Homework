\documentclass[
	%a4paper, % Use A4 paper size
	letterpaper, % Use US letter paper size
]{jdf}

\addbibresource{references.bib}

\author{Omair Tariq Khan}
\email{okhan60@gatech.edu}
\title{Homework 2}

\begin{document}
%\lsstyle

\maketitle
\hfill \break
\hfill \break


\section{Answer to Question 1 - Cooking}
With time, cooking has become an instinctive activity rather than a task that I have to meticulously work through. When I was a beginner, all the details surrounding cooking from recipes, measuring materials, and even the arrangement of the kitchen required constant supervision. I have to admit that I found myself uncertain and hopeless around cooking. But now, as I have come to understand the intricacies of cooking, I am able to enjoy the more creative aspects of it, instead of focusing on negatives of it.

The interface in this scope encompasses, among the many, recipes as the guidelines that provide a more structured approach to combining the ingredients, including measuring mugs, spoons and kitchen scales for precision, and the ingredients along with the cooking implements such as knives, pans and electrical devices which are the basic components of the task. Initially, it took a huge amount of time in recipe interpretation, ingredient measurement, and adjusting the cooking time and temperature. I would read the recipe several times to make sure that I didn't miss anything, look up every term that I did not know, and question my techniques. For instance, chopping vegetables took a lot of time since I wasn't confident with my knife skills. It had been daunting to time dishes to all finish at the same time, and even for simple things, such as boiling pasta, I would commonly use timers. The kitchen itself, where utensils and ingredients were was an interface I had to navigate consciously. 

Now, my thought process while cooking has shifted significantly. Recipes are more of a reference than a strict guide. I have committed so many fundamental techniques to memory-roasting, baking-that I don't need to read the instructions anymore, line by line. The quantities are approximate; I have learned to eyeball it, and I can tell just how much seasonings a dish will take. I am not thinking of components, I am thinking about the meal and what it's going to be like with the flavors and textures combined. Knowing how to set up the kitchen means I don't have to look around for this or that; it's just smooth, and I can enjoy the creativity of cooking rather than fumbling with logistical stuff. All these came with practice and repetition: every time cooking, developing that sense of time, combination, taste, techniques that worked better and mistakes allowed me to make necessary changes or recipes known so that then changing ingredients works very easily. Where I live now, I can also buy the seasoning packet for a particular dish I want to cook from my home country. 

In order to make this interface invisible sooner, I would design a computational system that bridges the gap between beginner and experienced cook. A perfect tool would integrate interactive recipes with real-time guidance. For example, through voice commands, an app may guide users step by step through recipes; at any time, users could stop and ask questions. It allows the user to visually see how to do various techniques, depending on their level. Also, the integration of smart kitchen tools like scales that automatically measure ingredients and sensors observing the cooking process, would reduce cognitive load for a beginner. The application may also give tips based on the user's past cooking history-suggesting seasoning adjustments or flagging common mistakes.

This could help speed up the learning curve by freeing up the user to be more creative with cooking sooner. Less intimidating and intuitive mechanics would make people more confident in experimenting in the kitchen. Of course, practice is key, but an interface designed properly could make those early stages of learning smoother and more pleasant.
\newpage

\section{Answer to Question 2}
 
\newpage

\section{Answer to Question 3}
\subsection {}

\newpage

\section{Answer to Question 4}


\newpage

\section{References}

\printbibliography[heading=none]
\begin{enumerate}
    \item 
    \item 
    \item 

\end{enumerate}

\end{document}
