\documentclass[
	%a4paper, % Use A4 paper size
	letterpaper, % Use US letter paper size
]{jdf}

\addbibresource{references.bib}

\author{Omair Tariq Khan}
\email{okhan60@gatech.edu}
\title{Homework 2}

\begin{document}
%\lsstyle

\maketitle
\hfill \break
\hfill \break


\section{Answer to Question 1 - Cooking}
With time, cooking has become an instinctive activity rather than a task that I have to meticulously work through. Initially, I was a beginner and all the details of cooking were new to me and these included recipes, measuring materials, and of course, the arrangement of the kitchen which all I needed were constant indications. I must confess that I felt lost and utterly confused in the kitchen. But at the present time, the basic knowledge of cooking that I have gained has helped to learn the creative side of it first, instead of focusing on negatives of it.

The interface in this scope includes recipe as the suggestive norms that offer a more structured approach to the combination of the ingredients, among which the measuring mugs, spoons, and science kitchen scales are the tools you must find to ensure precision, not to mention the ingredients and the cooking implements like knives, pans, and this kind of figural things, which are the basic components of the task. Initially, it was a massive workload of time in recipe interpretation, ingredient measurement, and adjusting the cooking time and temperature for me. To ensure I didn't overlook anything, I used to read the recipe a couple of times, browse each term that was alien to me, and also be reluctant about my methods. E.g., chopping vegetables took me a long time because I was not sure about my knife skills. Timing dishes to finish at the same time, which was even hard for me, for example, boiling pasta I usually used timers was quite a daunting task. The kitchen, which consisted of utensils and ingredients that were located in it, was an interface that I used to navigate very carefully.

Now, my thought process while cooking has shifted significantly. Recipes are more of a reference than a strict guide. I have committed so many fundamental techniques to memory-roasting, baking-that I don't need to read the instructions anymore, line by line. The quantities are approximate; I have learned to eyeball it, and I can tell just how much seasonings a dish will take. I am not thinking of components, I am thinking about the meal and what it's going to be like with the flavors and textures combined. Knowing how to set up the kitchen means I don't have to look around for this or that; it's just smooth, and I can enjoy the creativity of cooking rather than fumbling with logistical stuff. All these came with practice and repetition: every time cooking, developing that sense of time, combination, taste, techniques that worked better and mistakes allowed me to make necessary changes or recipes known so that then changing ingredients works very easily. Where I live now, I can also buy the seasoning packet for a particular dish I want to cook from my home country. 

In order to make this interface invisible sooner, I would design a computational system that bridges the gap between beginner and experienced cook. A perfect tool would integrate interactive recipes with real-time guidance. For example, through voice commands, an app may guide users step by step through recipes; at any time, users could stop and ask questions. It allows the user to visually see how to do various techniques, depending on their level. The integration of smart kitchen tools like scales that automatically measure ingredients and sensors observing the cooking process, would also help reduce cognitive load for a beginner. The application can also offer tips such as the user's past cooking activities suggesting seasoning changes or the user making common mistakes. 

Thus, the user would gain the freedom to be more creative, and it would probably shorten the learning curve. If the mechanics are not intimidating and they are intuitive to use, people will feel confident in trying out new things in the kitchen. Although it is important to have lots of practices, an interface well-designed could help to make the beginning stages of learning more straightforward and more enjoyable.
\newpage

\section{Answer to Question 2 - Using a smartphone to make video calls }
Making video calls on a smartphone is a task that engages multiple human perceptions, including visual, auditory, and haptic feedback. Each of these types of perception plays a crucial role in ensuring that the user can effectively place and participate in a video call. Visual perception is central to video calls. The screen displays the video picture, allowing users to see the person they are communicating with. It also shows essential interface elements such as buttons to accept or decline a call, mute the microphone, or turn the camera on or off. Notifications for incoming calls or poor connection quality also rely heavily on visual cues. For example, when a call is incoming, the screen typically lights up, displaying the caller's name or number along with an accept and decline button. During a call, visual feedback such as the connection strength indicator or a red border around the screen when recording ensures users are informed about the call’s status.

Auditory perception is equally important. The ringtone is the sound that the user hears when there is an incoming call and the auditory feedback from the caller during the call. The notification tones for new messages are also present among the sounds that other sounds the call on holding or resumed tones are the connection beeps that are typical of dropped audio. With the help of these sounds, the user can figure out what is happening without looking at the screen too much, thus making the user more focused on the conversation.

Haptic perception is available, but it is more subtle. The ring for the incoming call is usually mixed with the vibrations which are tactile feedback and are helpful especially when there are too many sounds to hear the auditory cues. A few devices I have come across have vibration modes to indicate additional features such as muting the microphone or disconnections. The feedback is helpful, as it creates the user awareness of the success of his input without the need of a visual check.

In terms of the modalities and the improvement of the feedback through them, I think some new implementations can be explored. A more dynamic interface for visual perception could integrate real-time emotion detection or control using gestures. Likewise, the application could work on making the participants' video feed stand out or with slight indications of who is active to speak next improve communication. Also, the spatial audio need not be used to give a sense of being in a room full of people having a call. The sound of the voices could be arranged so as if they come from other corners of the room which would have a positive impact on interaction to a greater extent.  Moreover, for personal audio alerts that may operate in the way, they could inform the user about the battery status, which is almost dead and needs to be plugged in shortly. With respect to haptic perception, making the vibration system more interactive could be useful. For example, different vibration patterns could mean different call states; a long vibration for a missed call, or short, repetitive pulses indicating that a call is waiting. Haptics could also be used to indicate gestures. As an example, a quick vibration could confirm a swipe to mute or dismiss a call without requiring the user to glance at the screen.

Outside of these three senses, the proprioception or the sense of body orientation and movement, is another mode that could be used for video calls. A gyroscope is a device on a mobile phone that can sense the orientation of the user, and Bluetooth is used to connect the application and the phone. For example, if the user tilts the phone, the camera can be really changed or the user can control it using gestures such as a nod to accept a call or a shake to decline it. These small features would add more functionality, and the whole experience would be more intuitive and human-like.
\newpage

\section{Answer to Question 3}
\subsection {Tip 1: Emphasizing Essential Content While Minimizing Clutter}

A good example of an interface that violates this principle would be the homepage for a streaming service like Netflix. When I log in, I am presented with a home page filled to the brim with endless rows of content, autoplaying previews, advertisements for their newest releases, and even notifications of account updates. While the notion is to offer choice, the number of visual and audio prompts becomes bewildering in trying to find what I am seeking. I find myself becoming confused and, for a moment or two, forgetting what I wanted to look up. This violates the principle of emphasizing essential content and low clutter, as it's distracting and adds unnecessary cognitive load when trying to make a simple choice, such as picking a movie.

To redesign the interface, Netflix could clean up the home page by showing fewer rows at a time and stopping the auto-playing previews unless explicitly clicked. Instead of hitting the user with all those categories at once, they could dynamically personalize a single row of high-relevance recommendations based on the user's viewing habits. Also, adding clear sections with collapsible menus for genres or categories will let users see more at their own pace. Simplifying the layout and focusing on what the user wants to do would make the interface less overwhelming and allow users to find what they want quickly and easily.

\subsection{Tip 2: Giving the User Control of the Pace}

A clear example of this being ignored is in my experience using Apple’s Fitness app for tracking workouts. During a workout, it will occasionally spam me with achievement or progress notifications. It takes my focus away and sometimes makes me think there's something I need to do instead of just letting me know an update has occurred. That loss of control over the presentation of information, when and how it's delivered, can make the app more difficult to navigate during a session.

How this app could be redesigned: It could give the user a "workout mode" where all notifications are paused so that data or achievements would be shown only after completion of the session; even better would be a customizable setting to decide how and when updates appear. This way, I could give my undivided attention to my workout sans any distractions. Since the app handed control over pace to me, it would fare much better to my needs without those unnecessary intermissions—ultimately a user-friendly experience.

\subsection{Conclusion}

Both these examples show how poorly designed interfaces increase cognitive load, either by overloading the user or by making the user adapt to rigid pacing. Those can be redesigned to emphasize important content and afford the user greater control, therefore making them more intuitive and user-friendly. Small changes in layout and allowing for customizable pacing will go a long way toward improving the user experience, reducing frustration, and thus ultimately making the interfaces usable by a wider audience.



\newpage

\section{Answer to Question 4}
One interface from my everyday life that feels really intolerant of errors is my bank's online banking app. A specific issue I’ve run into is when transferring money between accounts or paying bills. The app doesn’t give much room for mistakes—once you enter the amount and hit “Submit,” the transfer happens almost immediately, and there’s no way to cancel or adjust it afterward. The error is so easy to make, especially if you’re rushing or accidentally type an extra zero, turning a $50 payment into $500. The penalty for this mistake can be pretty severe, as you might overdraft your account or need to contact customer service to fix the issue. It’s frustrating, and it feels like the app is more focused on completing transactions quickly than on preventing users from making costly errors.

To address this, constraints could be added to the app to reduce the chance of making these errors in the first place. For example, the app could include a warning if the entered amount exceeds a typical range based on past transactions or the current balance in the account. It could also force users to confirm the amount in a pop-up before submitting the transfer. These constraints would act as a safeguard, giving users a chance to catch mistakes before they happen. For something as sensitive as financial transactions, this kind of built-in check would go a long way in preventing errors.

Improved mappings could also help reduce errors in this app. For instance, instead of just showing fields for inputting numbers, the app could show a visual representation of the accounts involved in the transaction. A slider could be added to let users choose an amount, with the slider clearly showing the account balance on each end. This mapping would make it easier to see how much money is available and what effect the transaction will have, giving users a clearer understanding of what they’re doing. By making the relationship between actions and their outcomes more obvious, the app could guide users away from mistakes.

Finally, improving affordances could make the app more error-tolerant. For example, the “Submit” button could be redesigned to visually change when clicked, like turning into a “Processing” state for a few seconds before finalizing the transaction. During that time, users could cancel if they notice an issue. Additionally, affordances like a dedicated “Undo” button or an option to schedule the transaction for later instead of immediately processing it would give users more flexibility to recover from errors. These small changes would make the interface feel more forgiving and user-friendly, especially for tasks as important as managing finances.

By incorporating constraints, better mappings, and improved affordances, the banking app could significantly reduce errors while making the entire process feel safer and more intuitive. It’s not just about preventing mistakes but also about giving users the confidence that the app has their back, even if they slip up.

\newpage

\section{References}

\printbibliography[heading=none]
\begin{enumerate}
    \item 
    \item 
    \item 

\end{enumerate}

\end{document}
