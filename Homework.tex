\documentclass[
	%a4paper, % Use A4 paper size
	letterpaper, % Use US letter paper size
]{jdf}

\addbibresource{references.bib}

\author{Omair Tariq Khan}
\email{okhan60@gatech.edu}
\title{Homework 2}

\begin{document}
%\lsstyle

\maketitle
\hfill \break
\hfill \break


\section{Answer to Question 1 - Cooking}
Over time, cooking has moved from a deliberative process to an intuitive one. In the beginning, I was new to cooking and everything about it was foreign to me; recipes, measuring materials and spoons, as well as the kitchen layout were all instructions I needed repeated constantly. The kitchen felt alien and overwhelming. But now that I have acquired some rudimentary knowledge about cooking, I can focus on it’s creative side first.

The interface in this scope includes recipe as the suggestive norms that offer a more structured approach to the combination of the ingredients, among which the measuring mugs, spoons, and kitchen scales are the tools you must find to ensure precision, not to mention the ingredients and the cooking implements like knives, pans, and this kind of figural things, which are the basic components of the task. Initially, it was a massive workload of time in recipe interpretation, ingredient measurement, and adjusting the cooking time and temperature for me. To make sure I didn't miss anything, I used to read the recipe a few times, look up every word I hadn't known and be doubtful of my actions. E.g., it took me ages to cut vegetables because I wasn't sure how good my knife skills were. Timing dishes to get them ready at the same moment, which already seemed almost impossible for me (that's why I always set timers), was nothing but a terrifying experience. The kitchen, with all tools and ingredients in it, was an interface that I used to navigate very carefully.

Nowadays, my thinking process during cooking is drastically different. Recipes are more like inspiration than strict instructions. I have memorized so many basic techniques - roasting, baking - that I no longer need to read their description line by line. Quantities are approximate; I can just as well estimate how much spices a dish will take. Instead of thinking about components, I think about systems and what kind of meal will come out if you mix flavors and textures together in such-and-such way-it often comes really weirdly to me! And finally: if you know where everything is in your kitchen, you don't have to rummage for this or that thing; everything goes smoothly and you can use your creativity instead of dealing with practical issues caught red-handedly.  All of these came with practice and repetition: every time cooking, developing that sense of time, combination, taste, techniques that worked better and mistakes allowed me to make necessary changes or recipes known so that then changing ingredients works very easily. Where I live now, I can also buy the seasoning packet for a particular dish I want to cook from my home country. 

In order to make this interface invisible sooner would be if I were to design a computational system bridging the gap between an experienced cook and beginner. It will be an intermediating tool tapping into easier ways would also help. Like maybe interactive recipes with real-time feedback app via voice command guiding you through the recipes but at any point you can stop and ask questions. You are able to see how to do various techniques based on your level visually as well. Then having smart kitchen tools like scales that can automatically measure(no need looking for conversion table), sensors watching the process as well helps in reducing cognitive load for a beginner. In a more perfect world, the app could also give tips about your past cooking activities e.g. suggesting you changing seasoning or when it notices you making common mistakes. 

So yes, the user would be more free to being creative, and it probably would be faster too. I mean, if the mechanics are not scary and easy to understand people will feel comfortable in experimenting. As much as you need practice a well designed interface could make those first steps way easier and much more satisfying.
\newpage

\section{Answer to Question 2 - Using a smartphone to make video calls }
Making a video call on a smartphone is an activity that engages multiple human perceptions: visual, auditory, and haptic. Each type of perception has its role in supporting the user to establish and maintain a video call. Visually, the screen not only displays the video picture allowing users to see the person they are communicating with, but also provides access to functionality: buttons for accepting or declining an incoming call; muting the microphone; and turning video capture functionality on or off. Various types of notifications are also displayed visually on the screen — e.g., incoming calls or low-network quality notifications. For instance, when an incoming call arrives, most screens light up while showing the name/picture of the caller along with answer/decline buttons. In addition to purely functionality-related visualization mentioned above (which designers should consider hiding as discussed earlier under “Haptic”), various forms of feedback visualizations provide users with information about system status throughout a call.

Auditory perception is equally important. The ringtone is the sound that the user hears when there is an incoming call and the auditory feedback from the caller during the call. The notification tones for new messages are also present among the sounds that other sounds the call on holding or resumed tones are the connection beeps that are typical of dropped audio. With the help of these sounds, the user can figure out what is happening without looking at the screen too much, thus making the user more focused on the conversation.

Haptic perception exists, but it's weaker. The ring for the incoming call is usually mixed with the vibrations which are tactile feedback and are helpful especially when there are too many sounds to hear the auditory cues. A few devices I have come across have vibration modes to indicate additional features such as muting the microphone or disconnections. The feedback is helpful, as it creates the user awareness of the success of his input without the need of a visual check.

In terms of the modalities and the improvement of the feedback through them, I think some new implementations can be thought of. For visual perception, a more dynamic interface could use real-time emotion detection/control by gesturing. Also, the application might just work on making the participants video stand out or maybe with some small cues who is active to speak next to improve communication. Also, the spatial audio need not be used to give a sense of being in a room full of people having a call. The sound of the voices could be arranged so as if they come from other corners of the room which would have a positive impact on interaction to a greater extent.  Moreover, for personal audio alerts that may operate in the way, they could inform the user about the battery status, which is almost dead and needs to be plugged in shortly. With respect to haptic perception, making the vibration system more interactive could be useful. For example, different vibration patterns could mean different call states; a long vibration for a missed call, or short, repetitive pulses indicating that a call is waiting. Haptics could also be used to indicate gestures. As an example, a quick vibration could confirm a swipe to mute or dismiss a call without requiring the user to glance at the screen.

Outside of these three senses, the proprioception or the sense of body orientation and movement, is another mode that could be used for video calls. There is already a gyroscope in your mobile phone that senses your orientation, and e.g. Bluetooth connects the phone with the application. If you tilted it, you would really rotate the camera or just use gestures like nodding for accepting a call and shaking your head for declining it! Those few things would give us more options but most importantly make this whole thing feel more natural!

\newpage

\section{Answer to Question 3}
\subsection {Tip 1: Emphasizing Essential Content While Minimizing Clutter}

A good example of an interface that violates this principle would be the homepage for a streaming service like Netflix. When I log in, I am presented with a home page filled to the brim with endless rows of content, autoplaying previews, advertisements for their newest releases, and even notifications of account updates. While the notion is to offer choice, the number of visual and audio prompts becomes bewildering in trying to find what I am seeking. I find myself becoming confused and, for a moment or two, forgetting what I wanted to look up. This violates the principle of emphasizing essential content and low clutter, as it's distracting and adds unnecessary cognitive load when trying to make a simple choice, such as picking a movie.

To redesign the interface, Netflix could clean up the home page by showing fewer rows at a time and stopping the auto-playing previews unless explicitly clicked. Instead of hitting the user with all those categories at once, they could dynamically personalize a single row of high-relevance recommendations based on the user's viewing habits. Also, adding clear sections with collapsible menus for genres or categories will let users see more at their own pace. Simplifying the layout and focusing on what the user wants to do would make the interface less overwhelming and allow users to find what they want quickly and easily.

\subsection{Tip 2: Giving the User Control of the Pace}

A clear example of this being ignored is in my experience using Apple’s Fitness app for tracking workouts. During a workout, it will occasionally spam me with achievement or progress notifications. It takes my focus away and sometimes makes me think there's something I need to do instead of just letting me know an update has occurred. That loss of control over the presentation of information, when and how it's delivered, can make the app more difficult to navigate during a session.

How this app could be redesigned: It could give the user a "workout mode" where all notifications are paused so that data or achievements would be shown only after completion of the session; even better would be a customizable setting to decide how and when updates appear. This way, I could give my undivided attention to my workout without any distractions. Since the app handed the control over pace to me, it would fare much better to my needs without those unnecessary intermissions, ultimately a user-friendly experience.

\subsection{Conclusion}

Both these examples show how poorly designed interfaces increase cognitive load, either by overloading the user or by making the user adapt to rigid pacing. Those can be redesigned to emphasize important content and afford the user greater control, therefore making them more intuitive and user-friendly. Small changes in layout and allowing for customizable pacing will go a long way toward improving the user experience, reducing frustration, and thus ultimately making the interfaces usable by a wider audience.



\newpage

\section{Answer to Question 4}
One interface that I feel isn’t very tolerant when it comes to mistakes is Zelle. I use it every month to pay my brother for our phone bill, and while it’s easy, if you do mess up it sucks. For example, I have before accidentally clicked on the wrong contact in my phone while sending money and unfortunately sent the payment. Since Zelle does instant transactions. You can’t cancel or receive your money back once sent out. What pissed me off about this most was having to contact that person telling them they needed to send that money back and then wait until they actually send you your money back; While still needing to make the original convenient payment towards your bill. The thought process is almost instantaneous—click click—and cha-ching, all gone. It’s frustrating, and it feels like the app is more focused on completing transactions quickly than on preventing users from making costly errors. 

Constraints may also be considered to prevent such mistakes case from occurring in the first place and to make Zelle more user-friendly.  For example, the app could include a warning if the entered amount exceeds a typical range based on past transactions or the current balance in the account. It could require users to confirm the recipient’s details—like their email or phone number—before finalizing the transaction. A simple pop-up that says, “Are you sure you want to send \$XX to [Name] at [Email/Phone]?” would add a helpful layer of verification. This would make it harder to accidentally send money to the wrong person, giving me more confidence in my transactions. For something as sensitive as financial transactions, this kind of built-in check would go a long way in preventing errors.

Improved mappings could be helpful for not only making the UI more apparent but also for preventing users from feeling anxious or guilty. Currently, the send button is too direct, and it’s a black box in terms of what happens after one presses it. By using a step-by-step process in which each step (selecting the recipient, inputting the amount, and confirming ) is visually separate might help prevent confusion. A status bar could show users where they are and ensure that they are not rushing through because as soon as all three steps are “100\% complete”, they can click to finalize their transaction. By making the relationship between actions and their outcomes more obvious, the app could guide users away from mistakes.

Better affordances could make Zelle more apparent. For example: hide the “send” select button until all details have been entered correctly (and confirmed). Or add a “review transaction” select button instead of “send”. This way perhaps people would click on it before sending money out and double-check everything before finalizing transactions. Also, if it gave you something like a dedicated “Undo” button or an option to schedule the transaction for later (like receive and pay) rather than processing right away would give people greater possibility to recover from errors. I believe the app can be more liberal/forgiving with options there, especially when money is involved. By adding constraints, better (automatically enforced) mappings and better affordances you not only lower chances of making an error but also give users sense of security and that they’re in control intuitively which this app does quite opposite. You want to make users feel safe that even if something goes wrong it’s gonna be fine.


\newpage




\end{document}
