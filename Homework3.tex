\documentclass[
	%a4paper, % Use A4 paper size
	letterpaper, % Use US letter paper size
]{jdf}

\addbibresource{references.bib}

\author{Omair Tariq Khan}
\email{okhan60@gatech.edu}
\title{Homework 3}

\begin{document}
%\lsstyle

\maketitle
\hfill \break
\hfill \break


\section{Answer to Question 1 - Playing FIFA}
I have played FIFA for about 16 years, so I am very familiar with the mistakes and problems that have been a part of the game since before I played it. In FIFA, a common slip occurs when a player intends to pass the ball to a teammate but they mistakenly press a wrong button, or multiple buttons in total, resulting in an unintended action. For example, I want to make a short ground pass using the designated button but accidentally press the shoot button instead, causing the ball to be launched forward and the player imbalancing and slipping. This slip usually happens due to the fast pace of the game, where quick decision-making is required, and muscle memory does not always align with the player's intention. 

This has been a part of different FIFA versions or gaming consoles that I have had, and though it has had its improvements, such "slips" are still prevalent. One way to reduce this is by improving the button feedback system, such as adding a subtle visual cue or vibration to confirm the intended action before execution. I haven't seen an option for a vibration apart from when taking a penalty kick. I have personally felt for a while that the interface should understand what the player is intending to do. For example, if I am intending to pass the ball to my teammate from the center of the ground, there is a seldom chance that I will shoot towards goal from there. So, if I press the shooting button accidentally, the game shouldn't allow me to shoot. 

A common mistake in FIFA happens when a player attempts to take a shot but does not understand how much power to apply, leading to a missed opportunity, which I have been a guilty of many times. As a beginner, I used to believe that holding down the shoot button longer will result in a more accurate or controlled shot when, in reality, it leads to an overpowered attempt that flies over the goal. This mistake happens because FIFA uses a power gauge system that requires precise timing, and newer players may not fully grasp how different shot types work. To prevent this mistake, FIFA could implement a clearer on-screen indicator that visually represents shot power in real-time before execution. Additionally, an interactive tutorial that adjusts based on player performance could help users better understand shot mechanics, reinforcing the correct approach through guided practice rather than trial and error.

One challenging aspect of FIFA that is neither a slip nor a mistake is tackling. Even having played different generations of the game, tackling and defending have been my weakest points. Defensive play, especially when trying to dispossess an opponent, requires precise timing and positioning and I have often found myself holding the tackle button, or tackling too early, which halts your player's movement for a second, causing your opponent to go past you. Even other experienced players struggle with executing clean tackles because if a tackle is mistimed, it can result in a foul or the opposing player easily dribbling past. The challenge here is not about knowing which button to press or making an accidental input but rather mastering the mechanics of defensive engagements. FIFA’s tackling system demands practice, as factors like player momentum, defensive awareness, and reaction speed all play a role. Newer versions of FIFA have trainings dedicated to different defending scenarios. While tackling remains difficult, it adds depth to gameplay and ensures that defensive skills are just as important as attacking abilities.
\newpage

\section{Answer to Question 2}
 
\newpage

\section{Answer to Question 3}
\subsection {}

\newpage

\section{Answer to Question 4}


\newpage

\section{References}

\printbibliography[heading=none]
\begin{enumerate}
    \item 
    \item 
    \item 

\end{enumerate}

\end{document}
