\documentclass[
	%a4paper, % Use A4 paper size
	letterpaper, % Use US letter paper size
]{jdf}

\addbibresource{references.bib}

\author{Omair Tariq Khan}
\email{okhan60@gatech.edu}
\title{Homework 3}

\begin{document}
%\lsstyle

\maketitle
\hfill \break
\hfill \break


\section{Answer to Question 1 - Playing FIFA}
I have played FIFA for about 16 years, so I am very familiar with the mistakes and problems that have been a part of the game since before I played it. In FIFA, a common slip occurs when a player intends to pass the ball to a teammate but they mistakenly press a wrong button, or multiple buttons in total, resulting in an unintended action. For example, I want to make a short ground pass using the designated button but accidentally press the shoot button instead, causing the ball to be launched forward and the player imbalancing and slipping. This slip usually happens due to the fast pace of the game, where quick decision-making is required, and muscle memory does not always align with the player's intention. 

This has been a part of different FIFA versions or gaming consoles that I have had, and though it has had its improvements, such "slips" are still prevalent. One way to reduce this is by improving the button feedback system, such as adding a subtle visual cue or vibration to confirm the intended action before execution. I haven't seen an option for a vibration apart from when taking a penalty kick. I have personally felt for a while that the interface should understand what the player is intending to do. For example, if I am intending to pass the ball to my teammate from the center of the ground, there is a seldom chance that I will shoot towards goal from there. So, if I press the shooting button accidentally, the game shouldn't allow the shot to happen. 

A common mistake in FIFA happens when a player attempts to take a shot but does not understand how much power to apply, leading to a missed opportunity, which I have been a guilty of many times. As a beginner, I used to believe that holding down the shoot button longer will result in a more accurate or controlled shot when, in reality, it leads to an overpowered attempt that flies over the goal. This mistake happens because FIFA uses a power gauge system that requires precise timing, and newer players may not fully grasp how different shot types work. To prevent this mistake, FIFA could implement a clearer on-screen indicator that visually represents shot power in real-time before execution. Additionally, an interactive tutorial that adjusts based on player performance in the newer version helps users better understand shot mechanics, reinforcing the correct approach through guided practice rather than trial and error.

One challenging aspect of FIFA that is neither a slip nor a mistake is tackling. Even having played different generations of the game, tackling and defending have been my weakest points. Defensive play, especially when trying to dispossess an opponent, requires precise timing and positioning and I have often found myself holding the tackle button, or tackling too early, which halts your player's movement for a second, causing your opponent to go past you. Even other experienced players struggle with executing clean tackles because if a tackle is mistimed, it can result in a foul or the opposing player easily dribbling past. The challenge here is not about knowing which button to press or making an accidental input but rather mastering the mechanics of defensive engagements. FIFA’s tackling system demands practice, as factors like player momentum, defensive awareness, and reaction speed all play a role. Newer versions of FIFA have trainings dedicated to different defending scenarios. While tackling remains difficult, it adds depth to gameplay and ensures that defensive skills are just as important as attacking abilities.
\newpage

\section{Answer to Question 2}
In the study titled "Analysis of Attitudinal Components Towards Statistics Among Students," the researchers aimed to understand students' attitudes toward statistics and how these attitudes vary across different groups \textbf{[1]}. The null hypothesis (H0) posited that there are no significant differences in attitudes toward statistics among the various student groups analyzed. Conversely, the alternative hypothesis (H1) suggested that significant differences do exist in these attitudes between the groups.

In this study, the independent variable was the grouping of students, which could be based on factors such as academic discipline or year of study. This variable is nominal, as it categorizes students into distinct groups without any inherent order. The dependent variables were the components of students' attitudes toward statistics, which were measured through various scales assessing aspects like anxiety, confidence, and perceived usefulness. These are typically measured on interval scales, as the differences between scale points are meaningful, but there is no true zero point.

The researchers employed paired Analysis of Variance (ANOVA) tests to compare the attitudinal components across different groups. ANOVA is a statistical method used to determine if there are statistically significant differences between the means of three or more independent (unrelated) groups. In this context, the researchers were looking to see if the mean attitude scores differed significantly among the student groups. They found that certain components of attitudes toward statistics did indeed vary significantly between groups, indicating that factors such as academic discipline or year of study might influence students' perceptions of statistics.

The use of ANOVA was appropriate for this study because it is designed to compare the means of multiple groups to see if at least one differs significantly from the others. The key assumptions for ANOVA include independence of observations, normally distributed dependent variables for each group, and homogeneity of variances. The study design ensured independence by categorizing students into distinct groups. While the paper does not explicitly state tests for normality, the use of interval scales for measuring attitudes typically assumes an underlying normal distribution. Homogeneity of variances, which assumes that different groups have similar variances, is another assumption of ANOVA. The paper does not detail whether this was tested, but it is a standard practice to check this assumption when performing ANOVA. Therefore, assuming these conditions were met, the application of ANOVA in this study was suitable for analyzing differences in students' attitudes toward statistics across various groups.

\newpage

\section{Answer to Question 3}
In my opinion, the interface that presents its content superbly is undoubtedly Google Maps. I nearly use it daily, both when I am traveling to a new area and just checking the traffic conditions. The thing that makes it the most useful is the high realism of its representation of the world. The routes, the buildings, and everything else, including little things such as gardens and parks, that are referenced in a way that you associate them with are just a matter of being laid out in an explicit form. Traffic indicators are the most useful of all—when one road is red, it simply means that it is congested, whereas, in the case of green, I can drive without any concerns. I have never had to "learn" how to use Google Maps, as the whole setup is completely intended to be in accordance with my mind's navigation process. This very thing can be considered a great example of the usability that design brings to communicating relationships. 

I can determine which roads intersect, where traffic will perhaps lengthen my time of arrival, and sometimes, although I won't acknowledge it, my route crosses a gas station. Another thing I appreciate is how the app doesn’t overload me with unnecessary details. When I zoom out, I notice among all the roads and highways only the big ones. When I zoom in instead, the information is more detailed, for example names of restaurants or small streets. This system is really clever since it is based on how I concentrate on small things in my normal life. For instance, I do not have to see all the street names in order to find a way to my destination. In this way, the interface of the app is designed effortlessly without having to think too much about it.

One particular user interface that really frustrates me is the setup option on my smart TV. I can no longer keep track of the numerous times that I have tried to do something as easy as turning off the motion smoothing and unexpectedly found that I had to go through a variety of menus instead. For instance, it is quite natural to expect picture settings such as brightness and contrast to be under the same setting, but on my TV, it is totally different; some of these settings are found under "Display" while others are placed at the very end in "Advanced Settings." This is simply illogical. A bad representation is exposed here clearly as it fails to show the relationships between settings. Instead, I am having to secondly scan settings and options that I do not even understand in order to get the one I am actually looking for. The only recourse left is to Google it, and even then, the descriptions I get are very complicated sometimes. A brief explanation like "reduces motion blur but may darken the screen" would be so much better. Conversely, the menu is crammed with ambiguous terms that adjusting my TV is far more complicated than it ought to be. This violates another key rule of good representation—it should avoid unnecessary complexity. The menu, for example, provides everything but the needed knowledge where a user such as me ends up with the obligation of researching just to set up basic settings. 

A user interface like Google Maps is very easy to use due to its intuitive design. Its functions appear to mirror the way I think a lot of the time. The absence of having any mental doubts about different colors representing some things or not being able to find key information quickly is very helpful-it is all logic and straightforward to me. In contrast, a badly designed interface, like my TV's settings menu, makes simple tasks harder than they need to be. The software doesn't help me to go through the process, but instead, it makes me to look for the answers elsewhere, which is very annoying. This is a great example to tell how crucial the good representations are. They are used so well that you don't notice them at all. Yet, when they are not, they turn even simple things into a chore. 

\newpage

\section{Answer to Question 4}
When I first tried logging into my Georgia Tech email through Duo’s single sign-on, I ran into an unexpected issue. I hadn’t actually set up Duo for my Georgia Tech account yet, so when I tried to log in, I was caught off guard by the fact that it was asking for a passcode. I had previously used Duo with my undergraduate institution, and from what I remembered, it would just send me a push notification to tap and approve the login. I had completely forgotten that there was a passcode option. After realizing that I needed to finish setting it up for my Georgia Tech account, I went through the process of linking my phone and generating the necessary codes. This step required me to understand where to go in the settings and how to navigate the Duo interface, but luckily, the setup instructions were clear enough. However, I can see how someone who isn’t as familiar with Duo might get confused if they were expecting only a push notification and suddenly had to enter a code instead. 

Logging into my Georgia Tech account using Duo's two-factor authentication with the code entry method involves several steps. First, I navigate to mail.gatech.edu in an incognito browser window and enter my username and password. This requires prior knowledge of my login credentials, which I have memorized from previous use. After submitting these details, the system prompts me to complete a two-factor authentication process. At this point, I need to open the Duo Mobile app on my smartphone, where I receive a notification indicating a pending authentication request. Within the app, I select the option to generate a passcode, which appears as a six-digit number. I then return to the browser finalize the login process. 

Evaluating this process through the lens of usability heuristics reveals both strengths and areas for improvement. One pertinent heuristic is the "Visibility of System Status," which emphasizes keeping users informed about what is happening through timely feedback. Duo excels in this regard by displaying real-time notifications on my smartphone, indicating when a login attempt is in progress and when a passcode is generated. Another relevant heuristic is the "Match Between the System and the Real World," which suggests that systems should use language and concepts familiar to users. Duo's use of terms like "passcode" and "authentication" aligns well with common security terminology, making the process intuitive. The app's interface mimics real-world token generators, displaying a numeric code that I can easily transcribe. 

However, the heuristic concerning "User Control and Freedom" highlights a potential limitation. If I mistakenly dismiss the authentication notification or close the app prematurely, there isn't an immediately obvious way to retrieve the current passcode or restart the process without initiating a new login attempt. From our lectures, I would also point out Consistency and Feedback as other heuristics. Consistency is important because users develop expectations based on past experiences. Since I was used to Duo automatically sending a push notification, I was thrown off when I saw a passcode instead. This inconsistency could be frustrating for users who assume that Duo always functions the same way. The interface would benefit from a clear indication of which authentication methods are available and why a passcode is needed to setup (at least the first time) instead of a push. Feedback is another crucial factor.  Duo does indicate that a passcode is sent to your phone,  there is small pop-up explaining that they need to open the Duo app and retrieve the code which makes the process smoother.

In summary, Duo's two-factor authentication process for Georgia Tech accounts demonstrates strong adherence to usability principles like visibility of system status and alignment with real-world conventions.  it could benefit from enhancements in user control and freedom. I haven't tried it, but I believe you would need a working internet to receive a push and setup your passcode in the app, which can be a bit problematic if you don't have access to the internet from your phone at that moment. 

\newpage

\section{References}

\printbibliography[heading=none]
\begin{enumerate}
    \item \textbf{Question 2}: \hfill \break
    Leon-Mantero C., Casas-Rosal J., Maz-Machado A., Rico M (2020, January). Analysis of attitudinal components towards statistics among students from different academic degrees
     

\end{enumerate}

\end{document}
