\documentclass[
	%a4paper, % Use A4 paper size
	letterpaper, % Use US letter paper size
]{jdf}

\addbibresource{references.bib}

\author{Omair Tariq Khan}
\email{okhan60@gatech.edu}
\title{Individual Project Check-in 1}

\begin{document}
%\lsstyle

\maketitle
\hfill \break
\hfill \break


\section{Needfinding Plan}
For this project, I will conduct two needfinding activities to gather insights into user needs and usability issues. The first activity will involve direct user interaction through a survey and interviews. The survey will target around 20 participants, each spending approximately 10 minutes, while interviews will involve five participants, each lasting about 20 minutes. The participants will primarily be classmates incentivized through course participation requirements, though I will also seek real users outside the class. The recruitment process will involve direct outreach and online postings. The survey will collect quantitative data on user habits, preferences, and pain points, while interviews will provide deeper qualitative insights.

The second needfinding activity will be a heuristic evaluation of an existing fitness app interface. I will use three heuristics derived from Nielsen’s ten usability heuristics: (1) Visibility of system status, (2) Match between the system and the real world, and (3) Aesthetic and minimalist design. I will evaluate how well the interface adheres to these principles by navigating through its key features, identifying usability issues, and documenting findings in a structured format. The evaluation will be conducted independently before synthesizing observations.

Both needfinding activities will provide a comprehensive understanding of user needs and usability issues, guiding the design process effectively.
\newpage

\section{Needfinding Results }
Making a video call on a smartphone is an activity that engages multiple human perceptions: visual, auditory, and haptic. Each type of perception has its role in supporting the user to establish and maintain a video call. Visually, the screen not only displays the video picture allowing users to see the person they are communicating with, but also provides access to functionality: buttons for accepting or declining an incoming call; muting the microphone; and turning video capture functionality on or off. Various types of notifications are also displayed visually on the screen — e.g., incoming calls or low-network quality notifications. For instance, when an incoming call arrives, most screens light up while showing the name/picture of the caller along with answer/decline buttons. In addition to purely functionality-related visualization mentioned above (which designers should consider hiding as discussed earlier under “Haptic”), various forms of feedback visualizations provide users with information about system status throughout a call.

Auditory perception is equally important. The ringtone is the sound that the user hears when there is an incoming call and the auditory feedback from the caller during the call. The notification tones for new messages are also present among the sounds that other sounds the call on holding or resumed tones are the connection beeps that are typical of dropped audio. With the help of these sounds, the user can figure out what is happening without looking at the screen too much, thus making the user more focused on the conversation.

Haptic perception exists, but it's weaker. The ring for the incoming call is usually mixed with the vibrations which are tactile feedback and are helpful especially when there are too many sounds to hear the auditory cues. A few devices I have come across have vibration modes to indicate additional features such as muting the microphone or disconnections. The feedback is helpful, as it creates the user awareness of the success of his input without the need of a visual check.

In terms of the modalities and the improvement of the feedback through them, I think some new implementations can be thought of. For visual perception, a more dynamic interface could use real-time emotion detection/control by gesturing. Also, the application might just work on making the participants video stand out or maybe with some small cues who is active to speak next to improve communication. Also, the spatial audio need not be used to give a sense of being in a room full of people having a call. The sound of the voices could be arranged so as if they come from other corners of the room which would have a positive impact on interaction to a greater extent.  Moreover, for personal audio alerts that may operate in the way, they could inform the user about the battery status, which is almost dead and needs to be plugged in shortly. With respect to haptic perception, making the vibration system more interactive could be useful. For example, different vibration patterns could mean different call states; a long vibration for a missed call, or short, repetitive pulses indicating that a call is waiting. Haptics could also be used to indicate gestures. As an example, a quick vibration could confirm a swipe to mute or dismiss a call without requiring the user to glance at the screen.

Outside of these three senses, the proprioception or the sense of body orientation and movement, is another mode that could be used for video calls. There is already a gyroscope in your mobile phone that senses your orientation, and e.g. Bluetooth connects the phone with the application. If you tilted it, you would really rotate the camera or just use gestures like nodding for accepting a call and shaking your head for declining it! Those few things would give us more options but most importantly make this whole thing feel more natural!

\newpage

\section{Answer to Question 3}
\subsection {Tip 1: Emphasizing Essential Content While Minimizing Clutter}

A good example of an interface that violates this principle would be the homepage for a streaming service like Netflix. When I log in, I am presented with a home page filled to the brim with endless rows of content, autoplaying previews, advertisements for their newest releases, and even notifications of account updates. While the notion is to offer choice, the number of visual and audio prompts becomes bewildering in trying to find what I am seeking. I find myself becoming confused and, for a moment or two, forgetting what I wanted to look up. This violates the principle of emphasizing essential content and low clutter, as it's distracting and adds unnecessary cognitive load when trying to make a simple choice, such as picking a movie.

To redesign the interface, Netflix could clean up the home page by showing fewer rows at a time and stopping the auto-playing previews unless explicitly clicked. Instead of hitting the user with all those categories at once, they could dynamically personalize a single row of high-relevance recommendations based on the user's viewing habits. Also, adding clear sections with collapsible menus for genres or categories will let users see more at their own pace. Simplifying the layout and focusing on what the user wants to do would make the interface less overwhelming and allow users to find what they want quickly and easily.

\subsection{Tip 2: Giving the User Control of the Pace}

A clear example of this being ignored is in my experience using Apple’s Fitness app for tracking workouts. During a workout, it will occasionally spam me with achievement or progress notifications. It takes my focus away and sometimes makes me think there's something I need to do instead of just letting me know an update has occurred. That loss of control over the presentation of information, when and how it's delivered, can make the app more difficult to navigate during a session.

How this app could be redesigned: It could give the user a "workout mode" where all notifications are paused so that data or achievements would be shown only after completion of the session; even better would be a customizable setting to decide how and when updates appear. This way, I could give my undivided attention to my workout without any distractions. Since the app handed the control over pace to me, it would fare much better to my needs without those unnecessary intermissions, ultimately a user-friendly experience.

\subsection{Conclusion}

Both these examples show how poorly designed interfaces increase cognitive load, either by overloading the user or by making the user adapt to rigid pacing. Those can be redesigned to emphasize important content and afford the user greater control, therefore making them more intuitive and user-friendly. Small changes in layout and allowing for customizable pacing will go a long way toward improving the user experience, reducing frustration, and thus ultimately making the interfaces usable by a wider audience.



\newpage

\newpage




\end{document}
