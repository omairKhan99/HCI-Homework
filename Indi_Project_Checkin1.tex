\documentclass[
	%a4paper, % Use A4 paper size
	letterpaper, % Use US letter paper size
]{jdf}

\addbibresource{references.bib}

\author{Omair Tariq Khan}
\email{okhan60@gatech.edu}
\title{Individual Project Check-in 1}

\begin{document}
%\lsstyle

\maketitle
\hfill \break
\hfill \break


\section{Needfinding Plan}
For this project, I will conduct two needfinding activities to gather insights into user needs and usability issues. The first activity will involve direct user interaction through a survey and interviews. The survey will target around 20 participants, each spending approximately 10 minutes, while interviews will involve five participants, each lasting about 20 minutes. The participants will primarily be classmates incentivized through course participation requirements, though I will also seek real users outside the class. The recruitment process will involve direct outreach and online postings. The survey will collect quantitative data on user habits, preferences, and pain points, while interviews will provide deeper qualitative insights.

The second needfinding activity will be a heuristic evaluation of an existing fitness app interface. I will use three heuristics derived from Nielsen’s ten usability heuristics: (1) Visibility of system status, (2) Match between the system and the real world, and (3) Aesthetic and minimalist design. I will evaluate how well the interface adheres to these principles by navigating through its key features, identifying usability issues, and documenting findings in a structured format. The evaluation will be conducted independently before synthesizing observations.

Both needfinding activities will provide a comprehensive understanding of user needs and usability issues, guiding the design process effectively.
\newpage

\section{Needfinding Results }
\subsection {Survey and Interview Findings}
The survey collected responses from 20 participants, providing insights into their fitness habits, app usage, and challenges. The data revealed that 75\% of respondents use a fitness app at least three times a week, primarily for tracking workouts and monitoring progress. However, 40\% expressed frustration with complex navigation and cluttered interfaces. Many users desired better personalization options, such as goal-specific recommendations and adaptive workout plans. Additionally, 65\% of respondents indicated that motivation and habit formation were key concerns, with most relying on notifications and progress tracking to stay engaged.

Interviews provided a deeper understanding of user motivations and pain points. Several participants emphasized the importance of social features, stating that competition and community support significantly impact their consistency. A recurring issue was the overwhelming nature of some fitness apps, where excessive features created confusion rather than clarity. One participant highlighted the need for a "quick start" feature, allowing users to log workouts without navigating through multiple screens. Another mentioned the difficulty of integrating fitness data from different sources, leading to fragmented tracking experiences.

\subsection {Heuristic Evaluation Results}
The heuristic evaluation of the fitness app revealed several usability concerns. Under the "Visibility of system status" heuristic, some key feedback mechanisms were either delayed or unclear. For example, after logging a workout, the confirmation message was subtle and easy to miss, leaving users uncertain whether their activity was recorded. Additionally, certain progress indicators lacked real-time updates, causing confusion when users expected immediate feedback on their workout completion.

For the "Match between the system and the real world" heuristic, the app presented some terminology that was unclear to novice users. Icons and labels were not always intuitive, leading to difficulty in navigating to essential features. Users unfamiliar with fitness tracking jargon struggled to interpret certain metrics, reducing the app’s accessibility for beginners.

Under the "Aesthetic and minimalist design" heuristic, the app contained an excessive number of elements on key screens, creating visual clutter. The home dashboard, in particular, displayed too much information at once, making it difficult for users to focus on their primary goals. Simplifying the interface and prioritizing essential features could improve usability.

Overall, the needfinding activities highlighted key areas for improvement, such as simplifying navigation, enhancing motivation features, and improving system feedback. These insights will directly inform design decisions, ensuring the final product aligns with user needs and usability principles.

\newpage

\newpage

\newpage




\end{document}
