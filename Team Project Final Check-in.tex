\documentclass[
	%a4paper, % Use A4 paper size
	letterpaper, % Use US letter paper size
]{jdf}
\usepackage{multirow}
\addbibresource{references.bib}

\author{Omair Tariq Khan, Danyal Ahmad, Atul Dhingra, Alice Maria Giani, Varun Basavaraj Kiragi}
\email{okhan60@gatech.edu, danyal@gatech.edu, adhingra43@gatech.edu , agiani3@gatech.edu, vkiragi3@gatech.edu}
\title{Team Project Final Submission \underline{\textbf{(YOU-ARE-NOT-YOUR-USER)}}}


\begin{document}
%\lsstyle

\maketitle
\begin{sloppypar}


\begin{abstract}
	This project aims to improve the user experience of applying for jobs through the Workday platform by addressing key usability challenges faced by applicants. Current issues include redundant data entry, lack of application status visibility, and inconsistent interface design across employer instances. Using human-centered design principles and iterative evaluation, the project seeks to develop a more intuitive and efficient application process. The goal is to create a solution that streamlines workflows, enhances transparency, and better supports the needs of modern job seekers.
\end{abstract}

\hfill \break

\section{Introduction \underline{(START OF CHECK-IN 1)}}
In today's rapidly evolving job market, the process of searching and applying for job opportunities has become increasingly complex and digitalized. Job seekers now routinely utilize online job portals, professional networking platforms, social media, and dedicated recruitment applications to identify potential career opportunities. This shift has increased accessibility and enabled individuals to explore a broader range of job options irrespective of geographical limitations. However, this digitalization has also introduced new complexities, with job seekers needing to monitor and apply through several different platforms, craft tailored digital resumes and profiles, and navigate algorithm-driven job recommendation systems.

Additionally, given the competitive nature of the contemporary job market, job seekers need to differentiate themselves from the other numerous applicants. Online platforms often involve automated screening processes that prioritize specific keywords, qualifications, and experiences, making it essential for candidates to understand these underlying selection mechanisms to maximize their visibility and chances of being shortlisted. Furthermore, the prevalence of virtual interviews and remote hiring practices has become increasingly common, requiring job seekers to develop additional competencies in digital communication and online presentation skills.

Among this extensive landscape of digital employment platforms aimed to streamline recruitment processes and simplify job applications and candidates selection, one popular app is Workday. Workday is a cloud-based software designed to integrate human resources and financial management processes, making it popular across several industries and widely used by companies. The app allows users to explore job opportunities, track multiple job applications, and manage their career data using a single app thanks to its comprehensive integration capabilities. Additionally, Workday ensures real-time updates and secure handling of sensitive data and personal information through advanced encryption and compliance with privacy standards. Not withstanding its positive features, Workday is often criticized for being counterintuitive, frustrating, and cumbersome for job seekers. Our project aims to reimagine and improve the Workday job application experience, making it more user-friendly, efficient, and responsive to the needs of applicants. The current system presents multiple pain points, including difficulties in saving progress, repetitive data entry, poor navigation, and a lack of transparency regarding application status.

Our goal is to redesign the job application process to be more seamless and intuitive. Job seekers should be able to easily apply for positions, track their progress, and receive relevant updates without unnecessary complexity. Many applicants, particularly those applying to multiple positions, experience frustration due to redundancies in form-filling and unclear user flows. Our project seeks to address these issues by identifying key user pain points, analyzing existing solutions, and designing a system that prioritizes efficiency and user satisfaction.

Through needfinding, we will gather insights from real users who have applied for jobs via Workday. We will examine their experiences, identify common frustrations, and use these insights to guide our design improvements. 

\newpage

\section{Needfinding Plan}
\subsection {User Interviews}
One of the most effective ways to gather firsthand insights is through direct user interviews. Our goal is to interview at least five individuals who have applied for jobs through Workday. These participants will be recruited through personal networks, online communities, and classmates who have used Workday for job applications. We will conduct 20-minute semi-structured interviews, asking participants about their experiences, challenges, and desired improvements.

Our questions will focus on:
\begin{itemize}
    \item The overall ease or difficulty of using Workday for job applications
    \item The most frustrating aspects of the process
    \item Any workarounds they have developed to navigate the system
    \item Features they wish existed to improve their experience
\end{itemize}

Potential biases in this method include self-selection bias (since only individuals willing to be interviewed will participate) and recall bias (participants may forget specific details). To mitigate these biases, we will ensure diversity in our participant pool and encourage detailed responses by asking follow-up questions.
\hfill \break

\subsection {Online Survey}
To complement our interviews, we will deploy an online survey targeting approximately 20 participants. The survey will gather quantitative and qualitative data about job application experiences across various platforms, with a specific focus on Workday.
The survey will include a mix of multiple-choice and open-ended questions covering:
\begin{itemize}
    \item Frequency of job applications
    \item Platform usability ratings
    \item Specific pain points in application processes
    \item Demographic information to contextualize responses
\end{itemize}

Potential biases in this method include self-reporting bias in interviews and surveys, and Potential interviewer influence. 

\hfill \break
\subsection{Heuristic Evaluation of Existing Workday Interface}
In addition to user feedback, we will conduct a heuristic evaluation of Workday’s job application system. Using usability heuristics such as Jakob Nielsen’s ten usability principles, we will systematically analyze the interface to identify usability issues. Three team members will independently evaluate the system and then compare findings to reduce individual bias.

The heuristics we will focus on include:
\begin{itemize}
    \item Visibility of system status (Are users informed about the progress of their application?)
    \item Match between system and real-world expectations (Does Workday use familiar language and workflows?)
    \item User control and freedom (Can users easily correct mistakes or navigate backward?)
    \item Consistency and standards (Are interactions intuitive and consistent?)
    \item Error prevention (Does Workday prevent common mistakes, such as duplicate applications?)
\end{itemize}
A potential bias in heuristic evaluation is subjectivity, as different evaluators may perceive issues differently. To counter this, we will discuss findings collectively and highlight only those issues that multiple evaluators agree on.

\hfill \break
\newpage
\section{Needfinding Results }
After executing our needfinding plan, we gathered valuable insights into the flaws of the Workday job application experience. Below, we summarize the key findings from each method and highlight the most significant takeaways.

\subsection{User Interviews}
We conducted five interviews with job seekers who have used Workday. Each participant expressed frustrations with the system, particularly regarding redundancy and poor navigation. Specific complaints included:
\begin{itemize}
    \item The need to manually enter the same information multiple times, even when uploading a résumé
    \item Difficulty in saving progress, leading to lost applications when switching devices
    \item Lack of clear feedback on application status after submission
\end{itemize}
\hfill \break

\subsection{Online survey}

We reviewed over 15 online complaints about Workday’s job application system. The most common issues mentioned were:
\begin{itemize}
\item Repetitive form-filling that does not respect uploaded résumés
\item Unclear communication regarding next steps after applying
\item Frequent login timeouts and session expirations causing lost progress
\end{itemize}

One user wrote, “I dread seeing ‘Apply with Workday’ on job postings because I know it means spending 30 minutes filling out forms that should be auto-filled.”

One participant mentioned that they often avoid applying for jobs through Workday due to how tedious the process is, preferring alternative platforms where available. Another noted that Workday’s interface is not mobile-friendly, making it frustrating to apply on a phone.
\hfill \break

\subsection{Heuristic Evaluation}
Our heuristic evaluation confirmed many of the usability issues identified in the interviews. The main problems included:
\begin{itemize}
\item Poor visibility of system status: Users do not receive adequate feedback on the status of their applications.
\item Inconsistent design: Different companies using Workday customize their portals, leading to a lack of uniformity in the application experience.
\item Redundancy is a primary frustration. Users strongly dislike entering the same information multiple times. Many expect résumé parsing but find that Workday does not adequately auto-fill fields.
\end{itemize}


\newpage

\section{Initial Brainstorming Plan \underline{(START OF CHECK-IN 2)}}
Our needfinding phase indicated significant user pain points with Workday's job application redundancy, limited transparency, and inadequate session management. Based on these insights, we planned a hybrid team brainstorming approach with two phases: asynchronous individual brainstorming followed by synchronous group discussion.

We will conduct a two-phase brainstorming process: First, we'll conduct individual ideation sessions where each team member generates ideas independently based on the needfinding insights. This approach helps mitigate groupthink by ensuring ideas are developed without immediate influence from others. Each member will document their ideas using a shared digital whiteboard, categorizing them according to the key pain points identified: redundant data entry, poor system visibility, inconsistent design, and mobile usability issues.

For the second stage, we'll conduct a synchronous group session using the "How Might We" framework to reframe problems as opportunities. We'll use prompts such as "How might we eliminate redundant data entry?" and "How might we improve application status visibility?" This technique encourages solution-oriented thinking rather than dwelling on problems. To mitigate anchoring bias, we'll rotate which team member presents their ideas first for each problem category.


\hfill \break
\newpage
\section{Brainstorming Results}
Our brainstorming session resulted in a range of ideas focused on improving the Workday job application experience. These ideas clustered around four main themes that aligned with our needfinding insights. The first theme centered on streamlining data entry, with ideas ranging from implementing AI-powered resume parsing to creating a universal Workday profile that persists across different employer instances. The second theme focused on improving transparency in the application process, including alerts for incomplete applications, upcoming deadlines, and interview schedules. The third theme addressed user interface consistency, proposing standardized design patterns and adaptive interfaces that maintain familiarity across different employer implementations. The fourth theme concentrated on improving the mobile experience and allowing users to save progress offline. From our pool of concepts, we selected three design alternatives to advance to the prototyping stage.

\subsection{Smart Profile}
The first is "Smart Profile," a centralized profile system that intelligently parses resume data and maintains user information across Workday instances. This directly addresses the frustration with redundant data entry that dominated our survey responses. Smart Profile envisions a system where users create a comprehensive profile once, which automatically populates application fields across all Workday instances. It incorporates machine learning to improve parsing accuracy over time and includes version control for different resume types. 

\subsection{Interactive Application Dashboard}
The second is the "Interactive Application Dashboard," a centralized hub where users can track applications, receive recommendations, and view upcoming interviews. It addresses the fragmented nature of the current Workday experience, which emerged as a key frustration in our need finding. Interactive Application Dashboard consolidates all application-related activities into a single, customizable interface where users can monitor application statuses, receive personalized job recommendations based on their profile and application history, and manage upcoming interviews or assessment tasks. This creates a cohesive experience that addresses the disjointed nature of the current system.

\subsection{Smart Application Flow}
The Smart Application Flow prototype reimagines the traditional job application process by making it more efficient, adaptive, and user-friendly. Rooted in insights from our needfinding research, this design shifts away from the rigid, linear application format in favor of a modular, card-based approach that enhances flexibility. By incorporating auto-save functionality, intelligent field pre-filling, and real-time validation, this prototype eliminates unnecessary friction while maintaining user control. It guides applicants through a seamless, interactive experience, minimizing redundant data entry while ensuring accuracy. This approach combines the best features of the Smart Profile and Interactive Application Dashboard, culminating in a more intuitive and accessible job application system.

\hfill \break
\hfill \break


\section{Initial Prototyping}
\subsection{Prototype 1: Smart Profile Prototype}
Our Smart Profile prototype addresses the widespread frustration with redundant data entry in Workday applications. The core feature is an intelligent resume parser that extracts detailed information from uploaded documents and maintains this data across employer instances.
The prototype presents a clean, intuitive interface with a prominent "Upload Resume" button at the center of the main profile page. Once a resume is uploaded, the system displays a visual representation of the parsing process, showing which sections are being analyzed (contact information, work experience, education, skills) with real-time progress indicators. After parsing completes, users are presented with their pre-filled profile, organized into collapsible sections that follow a logical hierarchy based on application importance.

The Smart Profile prototype leverages principles of recognition over recall by maintaining user data across sessions, reducing cognitive load when completing multiple applications. The design implements a clear visual hierarchy, with the most important actions (resume upload) prominently featured in a highlighted section. We incorporated progress indicators that show which profile sections are complete, providing immediate system status feedback.

The prototype addresses key usability heuristics identified in our needfinding. For instance, it solves the match between system and real world by intelligently parsing resume data into the appropriate fields, conforming to users' mental models. It also provides clear error prevention by validating information as it's entered and highlighting potential discrepancies between parsed resume data and existing profile information.

A particularly innovative feature is the version control system that allows users to maintain different profile versions for different job types, addressing the contextual nature of job applications highlighted in our interviews. The prototype includes a confidence indicator for parsed fields, allowing users to quickly identify and correct any parsing errors. The system also offers "smart suggestions" for improving profile completeness based on typical requirements for the user's target job categories.

The design incorporates a "privacy manager" feature where users can control which aspects of their universal profile are shared with specific employers, addressing privacy concerns raised during our needfinding interviews. Each section has clear edit controls with inline validation that provides immediate feedback, reducing form submission errors. The interface employs consistent visual design patterns throughout to establish a sense of familiarity and reduce learning curve—a direct response to the inconsistency issues identified in our heuristic evaluation.

\begin{figure}
    \centering
    \includegraphics[width=1.15\linewidth]{Smart Profile Prototype-1.png}
    \caption{Smart Profile}
    \label{fig:enter-label}
\end{figure}

\newpage

\subsection{Prototype 2: Interactive Application Dashboard Prototype}
The Interactive Application Dashboard prototype creates a centralized command center for job seekers, consolidating all application activities into a cohesive, personalized interface. This prototype directly addresses the fragmented experience cited by our survey respondents and all interview participants.

The dashboard employs a customizable grid layout with widget-like components that users can arrange according to their preferences. By default, it displays four primary sections: Active Applications, Recommended Jobs, Upcoming Tasks, and Recent Activity. Each section is visually distinct yet follows consistent design patterns, creating a unified experience that addresses the inconsistency issues identified in our heuristic evaluation.

The Active Applications section presents a card-based overview of all ongoing applications, with color-coded status indicators and prominently displayed upcoming deadlines or required actions. Each application card includes a miniature version of the Application Compass timeline, giving users an immediate understanding of their progress without requiring navigation to another screen. This implementation of the visibility of system status heuristic provides quick feedback on where users stand across multiple applications.

The Recommended Jobs section uses intelligent matching algorithms based on the user's profile, application history, and stated preferences to suggest relevant opportunities. Each recommendation includes a "match percentage" with a brief explanation of why it was suggested, creating transparency in the recommendation process. This feature directly addresses feedback from our needfinding where some of our respondents expressed frustration with irrelevant job recommendations.

The Upcoming Tasks section aggregates time-sensitive actions across all applications, such as scheduled interviews, assessments, or follow-up deadlines. Each task includes context about the related application and offers quick access to preparation resources. This implementation of the recognition over recall heuristic ensures users don't miss important deadlines due to the distributed nature of multiple applications.

The Recent Activity section provides a chronological feed of application updates, employer communications, and system notifications. This creates a comprehensive audit trail that helps users maintain awareness of their application landscape without having to check multiple sources—addressing the "black hole" feeling reported by many participants in our needfinding.

The dashboard incorporates contextual actions throughout the interface that change based on application status. For example, when an interview is scheduled, the system automatically offers calendar integration, preparation resources, and company research materials. This context-sensitivity creates a proactive system that anticipates user needs rather than requiring them to seek out information—directly addressing the lack of guidance cited in our survey responses.

A standout feature is the analytics panel that provides insights on application patterns, success rates across different job types, and comparative metrics with anonymized data from other job seekers. This data-driven approach empowers users with actionable information about their job search, addressing the feedback in our interviews that many users feel they're "applying blindly" without understanding effective strategies. The entire dashboard employs responsive design principles that maintain functionality across devices while optimizing the layout for different screen sizes. On mobile devices, the sections transform into a tabbed interface with critical notifications always visible, ensuring important updates aren't missed when using smaller screens, addressing the mobile usability concerns highlighted in our needfinding.


\begin{figure}
    \centering
    \includegraphics[width=1.1\linewidth]{Interactive2.png}
    \caption{Interactive Dashboard}
    \label{fig:enter-label}
\end{figure}
\newpage
\subsection{Prototype 3: Smart Application Flow}

The Smart Application Flow prototype addresses the inconvenient and inefficient application completion process that is evident from our needfinding research. For this prototype, we started from square one to truly understand how users fill out job applications and transformed it from a linear, cumbersome experience into a seamless, adaptive process.

The core design philosophy of this prototype is a modular, card-based user interface that partitions lengthy application forms into small, palatable sections. The Smart Application Flow presents content in a series of simplified steps. The interface includes a progress indicator that conveys the sections that have been completed and those remaining. Immediate feedback is provided to the user as they progress through the application, fulfilling the 'visibility of system status' heuristic mentioned earlier.

Interactive cards are used to present information to users. The sections can be expanded and collapsed, and users may freely travel between sections without losing progress. The prototype implements auto-save functionality, addressing a recurring pain point mentioned by several participants.

An integral feature of the Smart Application Flow is its contextual awareness. The prototype recognizes the information the user has previously provided in other applications and prefills those fields. An indicator shows the user which fields have already been filled in. Users can review these filled fields and modify as needed, which will then inform the program for future applications. This feature reduces the need for redundant data entry while balancing efficiency and control.

The interface uses dynamic validation to provide users with real-time feedback as they complete fields. Color-based indicators help alert users to issues before submission. Different colors are used to indicate different types of problems; this is to provide the user with additional insight instead of a binary valid/invalid.

The prototype provides intelligence assistance with examples and hints based on the specific role. For example, when filling out the work experience section, the system may suggest using particular terminology in the job description field. This helps align the user's context to the application process.

The "application drafts" feature enables users to save partially complete applications to finish later or use as a template. For example, users may create templates for different roles. This is in response to user feedback detailing the use of different application content based on the position being applied to.

To maximize user accessibility, the prototype implements a responsive design that adapts to the size of the device screen. On smaller screens, the card sections collapse into a vertical stack and adjust their height and width accordingly. This ensures that the interface works just as well on mobile devices.

A 'document locker' feature maintains a secure library of user pre-submitted application documents such as resumes, cover letters, and certificates. The locker maintains version control over these files and allows easy changes. The data from the documents is extracted and analyzed to prefill application sections and provide the user with key insights. The locker also reduces the need to re-upload the same documents for each application, thereby saving storage and network bandwidth.

The interface is built with a consistent design language throughout, with uniform patterns and actions. The consistency is intended to provide users with familiarity and minimize cognitive load from the application process.

The Smart Application Flow prototype integrates many of the insights gained from the Smart Profile and Interactive Application Dashboard prototypes. Combining the two, prototype three inches closer to a well-rounded interface. 

\begin{figure}
    \centering
    \includegraphics[width=1.1\linewidth]{SmartFlow1.png}
    \caption{Smart Flow}
    \label{fig:enter-label}
\end{figure}




















\newpage


\section{Evaluation Planning \underline{(START OF CHECK-IN 3)}}
Our evaluation plan aims to systematically compare our three prototypes: Smart Profile, Interactive Application Dashboard, and Smart Application Flow. We will evaluate these prototypes against each other and against the existing Workday interface to determine which design best addresses the pain points identified during needfinding.

\subsection{Participant Recruitment}

We will recruit 15 participants with prior experience using Workday for job applications. This sample size balances statistical significance with practical recruitment constraints. Participants will be recruited through:

\begin{itemize}
    \item Class participation pool (5 participants)
    \item Campus recruiting events (5 participants)
    \item Online job seeker communities (5 participants)
\end{itemize}

Participants will be invited based on their experience with job application systems and interest in contributing to better design. We will ensure diversity in participant demographics by recruiting individuals across different career stages (entry-level, mid-career, experienced professionals) and from various industries.

\subsection{Evaluation Process}

Each evaluation session will last approximately 45 minutes and follow a structured format:

\begin{enumerate}
    \item \textbf{Introduction and Background} (5 minutes)
    \begin{itemize}
        \item Brief overview of study purpose
        \item Demographic and background questions
        \item Consent form completion
    \end{itemize}
    
    \item \textbf{Prototype Interaction} (25 minutes)
    \begin{itemize}
        \item Sequential presentation of all three prototypes in randomized order
        \item For each prototype, participants will complete two standardized tasks:
        \begin{itemize}
            \item Task 1: Create a profile and submit a job application
            \item Task 2: Track application status and respond to an interview request
        \end{itemize}
        \item Think-aloud protocol during interaction
        \item Observer notes on navigation patterns and points of confusion
    \end{itemize}
    
    \item \textbf{Comparative Assessment} (10 minutes)
    \begin{itemize}
        \item Structured questionnaire comparing prototypes
        \item Preference ranking of prototypes
    \end{itemize}
    
    \item \textbf{Open-ended Discussion} (5 minutes)
    \begin{itemize}
        \item Qualitative feedback on overall experience
        \item Suggestions for improvement
    \end{itemize}
\end{enumerate}

\subsection{Data Collection Methods}

Our evaluation will collect both quantitative and qualitative data:

\subsubsection{Quantitative Metrics}
\begin{enumerate}
    \item \textbf{Task completion time} (measured in seconds)
    \item \textbf{Task completion success rate} (binary: completed/not completed)
    \item \textbf{Number of errors or backtracking instances} during task completion
    \item \textbf{System Usability Scale (SUS)} ratings for each prototype
    \item \textbf{Feature-specific satisfaction ratings} on a 7-point Likert scale:
    \begin{itemize}
        \item Ease of profile creation
        \item Clarity of application status
        \item Efficiency of data entry
        \item Intuitiveness of navigation
        \item Overall satisfaction
    \end{itemize}
\end{enumerate}

\subsubsection{Qualitative Data}
\begin{enumerate}
    \item \textbf{Think-aloud observations} during task completion
    \item \textbf{Semi-structured interview responses} about user experience
    \item \textbf{Observer notes} on participant behavior and verbal comments
    \item \textbf{Open-ended feedback} on strengths and weaknesses of each prototype
\end{enumerate}

\subsection{Data Analysis Plan}

\subsubsection{Quantitative Analysis}
\begin{itemize}
    \item We will conduct a one-way repeated measures ANOVA to compare task completion times across the three prototypes
    \item Post-hoc pairwise comparisons will use Bonferroni correction to account for multiple comparisons
    \item SUS scores will be calculated according to standard methodology and compared using paired t-tests
    \item Likert-scale satisfaction ratings will be analyzed using Friedman's test with follow-up Wilcoxon signed-rank tests
    \item Statistical significance will be determined at p < 0.05
\end{itemize}

\subsubsection{Qualitative Analysis}
\begin{itemize}
    \item Think-aloud data and interview responses will be transcribed and coded using thematic analysis
    \item Two team members will independently code the data to identify recurring themes
    \item Cohen's kappa will be calculated to ensure inter-rater reliability
    \item Themes will be organized into categories related to usability principles (efficiency, learnability, satisfaction, etc.)
    \item Observations will be triangulated with quantitative findings to provide contextual understanding
\end{itemize}

\subsection{Ethical Considerations}

To ensure ethical evaluation:
\begin{itemize}
    \item All participants will provide informed consent before participation
    \item Personal identifying information will be removed from collected data
    \item Participants may withdraw at any time without penalty
    \item Recordings will be stored securely and deleted after analysis is complete
\end{itemize}

This evaluation plan balances comprehensive assessment with practical implementation constraints, allowing us to systematically compare our prototypes on both objective performance metrics and subjective user experience dimensions.

\newpage
\section{Evaluation Results}

\subsection{Participant Demographics}

We successfully recruited 15 participants for our prototype evaluation. The sample consisted of:
\begin{itemize}
    \item 7 recent graduates/entry-level job seekers (0-2 years experience)
    \item 5 mid-career professionals (3-7 years experience)
    \item 3 experienced professionals (8+ years experience)
\end{itemize}

All participants had prior experience using Workday for job applications, with 80\% having used it within the past six months. Participants were recruited through class participation (6), campus events (5), and online communities (4). The gender distribution was 9 female and 6 male participants, with ages ranging from 22 to 45 years.

\subsection{Quantitative Results}

\subsubsection{Task Completion Times}

Task completion times were measured in seconds for each prototype:

\begin{table}[h]
\centering
\begin{tabular}{|l|c|c|}
\hline
\textbf{Prototype} & \textbf{Task 1 (Mean $\pm$ SD)} & \textbf{Task 2 (Mean $\pm$ SD)} \\
\hline
Smart Profile & 124.3 $\pm$ 18.5 & 76.2 $\pm$ 12.4 \\
Interactive Dashboard & 142.1 $\pm$ 22.7 & 58.5 $\pm$ 9.6 \\
Smart Application Flow & 105.8 $\pm$ 15.3 & 63.7 $\pm$ 10.8 \\
\hline
\end{tabular}
\caption{Task completion times in seconds for each prototype and task}
\end{table}

A one-way repeated measures ANOVA revealed significant differences in task completion times across prototypes (F(2,28) = 17.43, p < 0.001). Post-hoc tests showed that the Smart Application Flow prototype enabled significantly faster task completion for profile creation and application submission (Task 1) compared to the other prototypes (p < 0.01). For application tracking and interview response (Task 2), the Interactive Dashboard performed significantly better (p < 0.05).

\subsubsection{System Usability Scale (SUS)}

The SUS scores for each prototype (out of 100):

\begin{table}[h]
\centering
\begin{tabular}{|l|c|}
\hline
\textbf{Prototype} & \textbf{Mean SUS Score $\pm$ SD} \\
\hline
Smart Profile & 74.2 $\pm$ 8.7 \\
Interactive Dashboard & 76.8 $\pm$ 9.2 \\
Smart Application Flow & 83.5 $\pm$ 7.4 \\
\hline
\end{tabular}
\caption{System Usability Scale scores for each prototype}
\end{table}

Paired t-tests revealed that the Smart Application Flow had significantly higher usability ratings compared to both Smart Profile (t(14) = 3.86, p = 0.002) and Interactive Dashboard (t(14) = 2.75, p = 0.016).

\subsubsection{Feature-Specific Satisfaction Ratings}

Participants rated each prototype on a 7-point Likert scale for specific features:

\begin{table}[h]
\centering
\begin{tabular}{|l|c|c|c|}
\hline
\textbf{Feature} & \textbf{Smart Profile} & \textbf{Interactive Dashboard} & \textbf{Smart Application Flow} \\
\hline
Ease of profile creation & 5.3 $\pm$ 1.1 & 4.7 $\pm$ 1.2 & 6.1 $\pm$ 0.8 \\
Clarity of application status & 4.5 $\pm$ 1.3 & 6.4 $\pm$ 0.7 & 5.2 $\pm$ 1.0 \\
Efficiency of data entry & 5.8 $\pm$ 1.0 & 4.3 $\pm$ 1.4 & 6.2 $\pm$ 0.7 \\
Intuitiveness of navigation & 4.9 $\pm$ 1.2 & 5.6 $\pm$ 1.1 & 5.9 $\pm$ 0.9 \\
Overall satisfaction & 5.2 $\pm$ 1.0 & 5.5 $\pm$ 1.2 & 6.3 $\pm$ 0.8 \\
\hline
\end{tabular}
\caption{Feature-specific satisfaction ratings (mean $\pm$ SD) on a 7-point Likert scale}
\end{table}

Friedman's test indicated significant differences across prototypes for all features (p < 0.05). Follow-up Wilcoxon signed-rank tests showed that:
\begin{itemize}
    \item Smart Application Flow rated highest for ease of profile creation, efficiency of data entry, and overall satisfaction (p < 0.01)
    \item Interactive Dashboard rated highest for clarity of application status (p < 0.001)
    \item Smart Application Flow rated highest for intuitiveness of navigation, though not significantly different from Interactive Dashboard (p = 0.08)
\end{itemize}

\subsubsection{Prototype Preference Ranking}

When asked to rank prototypes in order of preference:
\begin{itemize}
    \item 60\% (9 participants) ranked Smart Application Flow as their first choice
    \item 27\% (4 participants) ranked Interactive Dashboard as their first choice
    \item 13\% (2 participants) ranked Smart Profile as their first choice
\end{itemize}

\subsection{Qualitative Results}

Thematic analysis of think-aloud protocols, observer notes, and interview responses yielded several key themes:

\subsubsection{Smart Profile Strengths}
\begin{itemize}
    \item Participants appreciated the intelligent resume parsing: \textit{"It was satisfying to see my resume accurately parsed without having to re-enter everything."}
    \item The version control for different job types was highlighted as innovative: \textit{"I like that I can maintain different versions for different types of jobs."}
\end{itemize}

\subsubsection{Smart Profile Weaknesses}
\begin{itemize}
    \item Some participants found the confidence indicators confusing: \textit{"I wasn't sure what to do when it showed low confidence for my skills section."}
    \item Privacy controls felt overwhelming to some users: \textit{"There were too many settings to configure before I could start applying."}
\end{itemize}

\subsubsection{Interactive Dashboard Strengths}
\begin{itemize}
    \item The visualization of application status was highly praised: \textit{"I could immediately see where I was in the process for each job."}
    \item Recommended jobs feature received positive feedback: \textit{"The match percentages helped me prioritize which jobs to look at next."}
\end{itemize}

\subsubsection{Interactive Dashboard Weaknesses}
\begin{itemize}
    \item Information density was concerning to some participants: \textit{"There's a lot happening on one screen, which is a bit overwhelming at first."}
    \item Some users wanted more customization: \textit{"I wish I could further customize which widgets appear where."}
\end{itemize}

\subsubsection{Smart Application Flow Strengths}
\begin{itemize}
    \item The card-based modular approach was universally appreciated: \textit{"Breaking the application into manageable chunks made it feel less daunting."}
    \item Auto-save functionality received enthusiastic feedback: \textit{"The auto-save feature gave me confidence that I wouldn't lose my work."}
    \item Contextual assistance was highlighted: \textit{"The suggestions for how to phrase my experience were really helpful."}
\end{itemize}

\subsubsection{Smart Application Flow Weaknesses}
\begin{itemize}
    \item Some participants wanted clearer indication of which sections were mandatory: \textit{"I wasn't sure which cards I absolutely needed to complete."}
    \item A few users noted that the document locker could be more prominently featured: \textit{"I almost missed the document locker feature."}
\end{itemize}

\subsubsection{Cross-Prototype Observations}
\begin{itemize}
    \item Participants consistently valued features that reduced redundant data entry
    \item Clear visual feedback on progress was important across all prototypes
    \item Mobile responsiveness was a critical factor for younger participants
    \item Participants often mentioned wanting to combine elements from different prototypes
\end{itemize}

\subsection{Summary of Findings}

The evaluation results revealed that the Smart Application Flow prototype performed best overall, with significantly higher usability scores and user preference ratings. Its modular, card-based approach to application completion addressed core pain points identified in our needfinding, particularly around reducing the feeling of overwhelm associated with lengthy application forms.

The Interactive Dashboard excelled specifically at providing clear application status visibility and job recommendations, while the Smart Profile was most effective at efficient profile creation and data entry. These findings suggest that an ideal solution might incorporate elements from all three prototypes, with the Smart Application Flow serving as the primary foundation due to its superior overall performance.

Quantitative and qualitative results were largely aligned, with features that received high satisfaction ratings also generating positive comments during interviews. The consistency across measurement approaches strengthens our confidence in these findings and provides clear direction for our second iteration.

\newpage
\section{Second Iteration Planning}

The goal of our project is to improve the user experience of applying for jobs through the Workday platform, which is a very common but often criticized job application platform. Workday’s interface often lacks clarity, personalization, and transparency in the job application workflow, making it a strong contender for getting a redesign using HCI principles. In our first design iteration, we came up with three different prototypes, which were Smart Profile, Interactive Application Dashboard, and Smart Application Flow. Each of these prototypes aimed to address a common user problem that was identified through our needfinding methods such as interviews, surveys, and heuristic evaluations. We chose to explore different design directions to further investigate how different types of interventions, like data reuse, visual guidance, and streamlined flow could help alleviate some of the major pain points in the Workday platform. After evaluating these prototypes, we found that the Smart Application Flow prototype outperformed the others in both usability and user preference. Our findings also showed that the users appreciated some features from our other prototypes that should be implemented further in future iterations. For the second iteration, we will base our high-fidelity prototype on parts from each of our three prototypes, and mainly base it off the Smart Application Flow prototype, since it received a lot of positive feedback from our user surveys. 

From the Smart Profile prototype, we will include some aspects such as resume parsing, profile version control, and smart suggestions for improving profile quality. From the Interactive Dashboard prototype, we will incorporate things like application status tracking, recommended jobs, and contextual alerts like interview reminders and deadlines. This synthesis of features reflects our goal of integrating user preferences while also addressing some overlooked usability gaps that we found in the initial designs. Based on our evaluation results, we spotted some new insights that will help us with our second iteration. The users valued features that minimized redundant data entry, like auto-filled fields and having support for templates. Application transparency and progress tracking were also requested, and they wanted features like visual timelines and clearer status updates. The participants liked the flow of the Smart Application Flow, but were a little unclear about the mandatory and optional fields. The privacy and version control options in the Smart Profile section received positive feedback as well, even though some of the users found them a little confusing at times. While each prototype addressed most of the user concerns, we do not have any single design that captures the complete application experience. Given the breadth of insights from our evaluation, we think that we have a good direction to go with for our second iteration.

In the next iteration, we will look deeper into some usability issues that we noticed from the evaluation. We will implement better visual indicators for required vs optional fields in the application flow to reduce the confusion. We will also make some changes to the document locker interface to improve visibility and ease of access. We will also consider some mobile interface responsiveness by adding more user friendly layouts and optimizing the layouts for different screen sizes. We also hope to test out how intuitive the new profile and workflow is in comparison to our previous prototypes. In the evaluation, we will test our high-fidelity prototype and focus on gathering both quantitative and qualitative data, focusing on task efficiency, system usability scores, satisfaction with new features, and overall ease of use. We will also measure the discoverability of features like resume templates and the document locker to ensure that they are integrated properly within our user interface. This evaluation will help us determine whether or not our prototype is an improvement and simplifies the application process while meeting the user needs. Overall, by combining the best aspects of each of our three prototypes and resolving the remaining issues, our second iteration will provide a more holistic, streamlined, and user-centric job application experience. Below is a table showing how all design goals align.

\begin{figure}
    \centering
    \includegraphics[width=0.75\linewidth]{Second Iteration Planning.jpg}
    \caption{Second Iteration Planning}
    \label{fig:enter-label}
\end{figure}


\newpage
\section{Final Prototype}
\subsection{Smart Application Flow 2.0}
Based on our evaluation results, the Smart Application Flow prototype received the highest usability scores and user preference ratings. For our final prototype, we've refined this design while incorporating key strengths from the other prototypes. Smart Application Flow 2.0 maintains the core card-based modular approach while adding enhanced features for profile management, application tracking, and contextual assistance.
The refined prototype addresses several areas of improvement identified in our evaluation:
\begin{itemize}
    \item Clear visual indicators for required vs. optional fields
    \item Enhanced document locker with improved visibility
    \item Responsive design optimized for mobile devices
    \item Integration of smart profile features from our first prototype
    \item Incorporation of status tracking elements from the Interactive Dashboard
\end{itemize}

Our final prototype is designed as a responsive web application with an emphasis on usability across device types. The prototype uses a consistent color scheme throughout, with blue as the primary action color, orange for alerts and notifications, and a clean white background for optimal readability. Typography follows accessibility guidelines with sufficient contrast ratios.
Key design decisions include:
\subsubsection{Modular Card-Based Application Interface}
The primary interface maintains the card-based approach that received positive feedback in our evaluation. We've enhanced this feature by implementing clearer visual indicators for required versus optional sections, addressing a key point of confusion identified in our evaluation. Cards are now color-coded with required sections featuring a distinct blue header and optional sections using a lighter gray header. This design decision implements the "visibility of system status" heuristic by making system requirements immediately apparent to users.

\subsubsection{Enhanced Profile Management with Smart Parsing}
Incorporating the most valued elements of the Smart Profile prototype, we've implemented an intelligent resume parser that extracts detailed information from uploaded documents and presents it in an editable format. Following the principle of user control and freedom, we display confidence indicators for each parsed field while allowing easy modification. This strikes the balance between automation and user control that participants valued during evaluation.

\subsubsection{Status Tracking Integration}
From the Interactive Dashboard, we've incorporated a simplified status tracking bar at the top of the application, providing users with immediate visibility into their application progress without the complexity that some users found overwhelming in the original dashboard design.

\subsubsection{Document Locker Enhancement}
We've made the document locker more prominent and intuitive, featuring it as a dedicated card in the application flow with visual previews of stored documents. This addresses feedback that the feature was useful but sometimes overlooked.

\subsubsection{Contextual Assistance with Smart Suggestions:}
We've refined the contextual hints to be less intrusive while still providing valuable guidance, showing suggestions only when users interact with specific fields rather than presenting all suggestions simultaneously.

\begin{figure}
    \centering
    \includegraphics{Dashboard Final Prototype-1.png}
    \caption{Final Dashboard Prototype}
    \label{fig:enter-label}
\end{figure}

\begin{figure}
    \centering
    \includegraphics{Application Final Prototype.png}
    \caption{Final Application Prototype}
    \label{fig:enter-label}
\end{figure}

\begin{figure}
    \centering
    \includegraphics{Profile Final Prototype-1.png}
    \caption{Final Profile Prototype}
    \label{fig:enter-label}
\end{figure}

\newpage




\section{Final Evaluation Planning}

Our final evaluation will focus on assessing the usability and effectiveness of the Smart Application Flow 2.0 prototype. We aim to gather both qualitative and quantitative feedback to determine if our design improvements address the key pain points identified in our initial evaluation.

\subsection{Evaluation Process Overview}

The evaluation will involve participants interacting with the high-fidelity prototype followed by a mixed-method assessment:

\begin{itemize}
    \item \textbf{Participants:} We will recruit 10-15 participants from our university's student population who have experience applying for jobs online. This includes both undergraduate and graduate students across different majors.

    \item \textbf{Recruitment Strategy:} Participants will be recruited through campus job boards, social media groups, and direct outreach to career services. 

    \item \textbf{Session Structure} (approximately 15 minutes per participant):
    \begin{itemize}
        \item Brief introduction (2 minutes)
        \item Guided interaction with prototype (7 minutes)
        \item Quantitative survey (3 minutes)
        \item Semi-structured interview (3 minutes)
    \end{itemize}

    \item \textbf{Evaluation Environment:} Sessions will be conducted remotely using Zoom, with screen sharing enabled to observe participant interactions with the prototype.
\end{itemize}

\subsection{Evaluation Methods}

The evaluation will employ both quantitative and qualitative methods to comprehensively assess the prototype:

\begin{enumerate}
    \item \textbf{Task-Based Evaluation:} Participants will complete three specific tasks:
    \begin{itemize}
        \item Task 1: Upload a resume and review auto-filled information
        \item Task 2: Complete the skills section using the card interface
        \item Task 3: Navigate between application sections while maintaining progress
    \end{itemize}

    \item \textbf{Quantitative Assessment:} Participants will complete a survey including:
    \begin{itemize}
        \item System Usability Scale (SUS) for standardized usability measurement
        \item Task-specific metrics (completion time, error rate)
        \item 7-point Likert scale ratings for specific features:
        \begin{itemize}
            \item Clarity of required vs. optional fields
            \item Ease of document management
            \item Helpfulness of smart suggestions
            \item Overall satisfaction
        \end{itemize}
    \end{itemize}

    \item \textbf{Qualitative Assessment:} Semi-structured interviews will explore:
    \begin{itemize}
        \item Most valuable aspects of the design
        \item Areas of confusion or frustration
        \item Perceived improvement over traditional job application experiences
        \item Suggestions for additional features or refinements
    \end{itemize}
\end{enumerate}

\subsection{Data Analysis Plan}

\begin{enumerate}
    \item \textbf{Quantitative Analysis:}
    \begin{itemize}
        \item Calculate average SUS score and compare against industry benchmarks (target: 80+)
        \item Analyze descriptive statistics for task completion times and error rates
        \item Examine feature-specific ratings to identify strengths and weaknesses
    \end{itemize}

    \item \textbf{Qualitative Analysis:}
    \begin{itemize}
        \item Conduct thematic analysis of interview responses
        \item Code feedback into categories (interface, functionality, content, etc.)
        \item Identify patterns in participant suggestions and pain points
        \item Triangulate with quantitative findings to develop holistic insights
    \end{itemize}
\end{enumerate}

The combination of these methods will provide a comprehensive understanding of the prototype's effectiveness and guide further refinements before implementation. Detailed evaluation materials including task instructions, survey questions, and interview guides are provided in \textbf{APPENDIX J}.

\newpage

\section{Video Prototype}
A video demonstration of the final prototype is available at:

\begin{center}
    \url{https://mediaspace.gatech.edu/media/t/1_hdib552z}
\end{center}

The prototype can be accessed here:

\begin{center}
    \url{https://da2853.github.io/You-Are-Not-Your-User/}
\end{center}

\newpage

\section{Final Evaluation Results}
We conducted our final usability evaluation with 15 participants, all of whom had experience with applying for jobs using the Workday application. All of our participants interacted with our final prototype in a remote settings. The group included a diverse mix of students and professionals, which helped ensure a good representative sample of our target user base. We designed this evaluation to assess both task performance and subjective usability, while also gathering deeper insights into how well users understood and discovered our new features. After each evaluation session, the participants completed the system usability scale (SUS). The average score for SUS was 85, which gave us an excellent rating and exceeded our expectations. This suggests that the users found our final prototype very usable, intuitive, and efficient for completing job applications and other related tasks. Participants responded very positively to nearly every feature. Users liked the ability to progress through the job application at their own pace, and did not feel overwhelmed by having too much content on the screen at once. The clean and simple user interface allowed the users to revisit and revise sections without losing their place, which was a common frustration that users had with the original Workday interface. The card flow system was particularly well-received. The users appreciated the ability to jump between sections without fear of losing progress, and the auto-save confirmation indicator was useful, and helped reassure them that their work was being preserved in a real-life scenario. Below is a figure that shows all of the completion time, error rates, and friction points across the three core tasks. None of the users failed any of the tasks, and the error rates were relatively minimal. The only recurring issue came up during the resume review, where some of the participants felt a little confused about which fields were editable after using the auto-fill feature.
\begin{figure}
    \centering
    \includegraphics[width=1\linewidth]{Screenshot 2025-04-15 at 4.49.07 PM.png}
    \caption{Findings}
    \label{fig:enter-label}
\end{figure}


Beyond the numerical data, we conducted some post-session interviews and analyzed the written feedback. The participants responses consistently reinforced the strengths of our design while still revealing some small areas for future improvement. Some of the participants said that breaking up the application process into smaller portions made the process less overwhelming and easier to digest. This feedback supports one of the core design goals we had in this project, which was to reduce cognitive overload by avoiding long, cluttered, and unaesthetic forms. The visual consistency across sections also contributed to smoother navigation. By using similar layout patterns and button placements throughout the prototype, users could predict interactions even when encountering a new feature for the first time. We also noticed that users quickly picked up on the navigation breadcrumbs, which were placed at the top of each section to provide context and help them understand where they were in the process. Instead of feeling burdened by having long forms that were not visually pleasing, the users said that the smaller steps made it easier to stay more motivated to complete all of the sections. The participants who used the resume enhancement tool also said it would be a useful feature. Many also appreciated seeing an indicator or "last saved" message during the application process. The document locker idea was also well received. Moving onto some future improvement suggestions, although most of the visuals were an improvement over the regular Workday application and the earlier iterations, some users still felt a little confused on what was required to fill out and what was optional. One participant noted that the progressive disclosure of content helped them stay focused on the current task. This was an unanticipated but encouraging sign that our interface could be more inclusive by design. We also received comments from users with different interface preferences, and received mentions that the step-by-step flow helped them stay on task more effectively. In addition to functional praise, several participants described the interface as calming and less intimidating than typical job portals. The balance between minimalism and control appeared to reduce user anxiety, especially when navigating through multiple form fields. One participant mentioned that they found the application process even somewhat enjoyable. Another participant mentioned that they appreciated the clean visual spacing and high contrast of the prototype's interface. Another participant suggested adding keyboard shortcuts or a condensed view for users who apply to many jobs in one session. Although this was not part of our current design, it presents an interesting opportunity for future iterations.  Some users also missed the document locker icon, and took them a couple of seconds to figure it out. Users would also prefer if they had a clearer confirmation that some fields had been updated after using the smart suggestions feature. By simulating realistic job application scenarios and having participants speak aloud during the process, we were able to capture not just what they did, but also how they felt. This methodology provided deeper insight into both conscious and subconscious usability issues. We intentionally included a wide variety of tasks, which range from uploading documents to reviewing suggestions to test the breadth of our prototype's utility. These tasks were modeled after common pain points reported in our initial needfinding phase. Our final evaluation gives us strong evidence that our design meaningfully improves the Workday job application experience. From further streamlining workflows to reducing friction points in data entry, our high-fidelity prototype resolved many of the key issues we identified during our earlier need-finding steps. Compared to our early prototpes and participants' experience with the default Workday interface, most users completed tasks noticeably faster and with fewer hiccups, which indicate some improvements in speed and confidence in the user interface. This project reaffirmed the importance of designing not just for task completion, but for user trust, flexibility, and peace of mind. Features like auto-save, version control, and modular flow all worked together to create an interface that felt both modern and easy to use. If we were to continue developing and iterating over this project, our next steps would include a quick and simple on-boarding process to highlight the key features we added in our project, as some participants requested guidance when first using the system. Overall, our final evaluation not only validated our core design decisions but also provided a roadmap for refining this prototype into a fleshed out solution that could be a viable improvement to the Workday platform. This project also demonstrated the value of iterative, evidence-based design. By continually testing and refining based on real user feedback, we were able to build a system that not only addressed known issues but also showed us new opportunities for improvement. 

\newpage

\section{Individual Reflections}
\subsection{Omair Tariq Khan}

Throughout this project, I contributed across all phases, especially in areas related to planning, evaluation, and documentation. In Check-in 1, I led the creation of the Needfinding Plan, wrote the Needfinding Results, and conducted the heuristic evaluation, consolidating and interpreting data from surveys (created by my teammate \textbf{Alice}) and interviews. I also handled the appendices for this section. For Check-in 2, I compiled the Initial Brainstorming Plan and proposed two of the three major ideas we carried forward: Smart Profile and Interactive Application Dashboard, with also creating the medium-fidelity prototype for Smart Profile. In Check-in 3, I fully developed the Evaluation Planning section and executed the Evaluation Results, including statistical analysis and summary. In the final stage, I initiated and refined the Final Prototype, and called it "Smart Application Flow 2.0" by integrating features based on our earlier findings and design iterations. I also wrote the Final Evaluation Planning section and compiled its appendix.

Each teammate brought something unique to the table. Alice helped with the initial Introduction and created the survey that fed into our needfinding. \textbf{Danyal} played a big role in prototyping—he designed the Interactive Dashboard, built out the Smart Application Flow, and refined and deployed the final version on GitHub. He also created the video walkthrough. \textbf{Varun} focused on the Second Iteration Planning, and his ideas were really useful when shaping the final prototype. He also developed the Final Evaluation Results. \textbf{Atul Dhingra} helped summarize the findings from our Final Evaluation Results that and pulled together the appendix for that part. 

Overall, the project went well due to our structured process, consistent communication, and shared understanding of user-centered design principles. Our iterative approach from needfinding to final evaluation ensured our decisions were data-driven and user-focused. The biggest strength was how insights from early stages flowed naturally into prototyping and evaluation, especially when our Smart Application Flow 2.0 prototype addressed nearly all the pain points we initially identified.That said, there were aspects of the project that could have been improved. Time management was occasionally a challenge, especially when coordinating across deliverables for each check-in.

\newpage

\subsection{Alice Maria Giani}

\newpage

\subsection{Muhammad Danyal}

\newpage

\subsection{Varun Basavaraj Kiragi}

\newpage

\subsection{Atul Dhingra}

\newpage

\section{Appendix}
\subsection{Appendix A: Needfinding Plan}

Our needfinding strategy is designed to comprehensively understand the job application experience through three complementary research methods:

\subsubsection{1. User Interviews}
\begin{itemize}
\item Target Participants: 5 individuals across different career stages
\item Duration: 20 minutes per interview
\item Focus Areas: Emotional experiences during job applications, Specific challenges in online application processes, Desired features and improvements
\end{itemize}
\hfill \break

\subsubsection{2. Online Survey}
Introduction for Survey Participants:
Thank you for participating in our survey! We are conducting research on job application experiences, specifically focusing on Workday. Your responses will help us identify common frustrations and potential improvements. The survey should take about 10 minutes to complete.

\underline{\textbf{Demographics \& Background}}

\textbf{1. SELECT YOUR AGE:}

18-29

30-39

40-49

50-64

65+

\hfill \break

\textbf{2. WHAT IS YOUR CURRENT EMPLOYMENT STATUS?}

Employed full-time

Employed part-time

Unemployed (seeking employment)

Unemployed (not seeking employment)

Student (seeking employment)

Student (not seeking employment)

Retired


\underline{\textbf{Usage  }}

\textbf{3. WHAT TYPE OF JOB(S) HAVE YOU APPLIED TO USING WORKDAY APP? (CHECK ALL THAT APPLY)
}

Internship

Entry-level

Mid-level

Senior-level

Executive roles \\



\textbf{4. WHEN SEARCHING FOR A JOB, HOW OFTEN DO YOU USE WORKDAY APP FOR JOB SEARCH AND APPLICATIONS?
}

Several times a day

Daily

More than 4 time per week

At least 2 times per week

Weekly

Monthly

Rarely

Never			
\\

\underline{\textbf{Application Experience  }}

\textbf{5. HOW WOULD YOU RATE YOUR OVERALL EXPERIENCE OF SEARCHING FOR JOBS  USING WORKDAY APP?}

Very satisfied

Satisfied

Neutral

Dissatisfied

Very dissatisfied \\



\textbf{6. HOW WOULD YOU RATE YOUR OVERALL EXPERIENCE OF APPLYING TO JOBS THROUGH WORKDAY?}

Very satisfied

Satisfied

Neutral

Dissatisfied

Very dissatisfied \\

\textbf{7. WHAT FEATURES OF WORKDAY DO YOU USE FOR JOB SEARCHES AND APPLICATIONS? (SELECT ALL THAT APPLY) }

Job search filters

Application submission

Resume upload

Profile updates

Application status tracking

Notifications

Others \\

\textbf{8. IF YOU HAVE SELECTED OTHERS, PLEASE SPECIFY BELOW
}

\textbf{9. HOW SATISFIED ARE YOU WITH THE OVERALL EASE OF USING WORKDAY?}

Very satisfied

Satisfied

Neutral

Dissatisfied

Very dissatisfied \\

\textbf{10. HAVE YOU ENCOUNTERED ANY OF THE FOLLOWING CHALLENGES WHEN WORKING WITH WORKDAY? (SELECT ALL THAT APPLY)
}

Difficulty finding jobs of interests

Difficulty uploading documents (resume, cover letters, etc.)

Confusion navigating the app interface (for example difficulty in going back, etc …)

Slow response/loading times

Session expiration issues

Lack of clear instructions or guidance

Errors or bugs when applying for jobs

Inconsistent application status notifications or updates

Editing an application

Others \\

\textbf{11. IF YOU HAVE SELECTED OTHERS OR WISH TO PROVIDE ADDITIONAL INFORMATION, PLEASE SPECIFY BELOW} 

\textbf{12. HOW FREQUENTLY DO YOU ENCOUNTER TECHNICAL ISSUES WHEN USING WORKDAY? 
}

Very Frequently
Frequently

Occasionally

Rarely

Never \\

\textbf{13. IF APPLICABLE, DESCRIBE ANY INSTANCES WHEN YOU HAVE ABANDONED A JOB APPLICATION THROUGH WORKDAY DUE TO ITS COMPLEXITY OR TECHNICAL ISSUES 
}

\underline{\textbf{Suggestions for improvement }}

\textbf{14. WHICH FEATURES WOULD YOU LIKE TO SEE IMPROVED IN WORKDAY? (SELECT ALL THAT APPLY)
}

Job search functionality

Uploading functionality

Instructions on how to use Workday

Navigation options

Auto-filling application from résumé  

Application status updates 

Mobile-friendly interface

Updates and notifications

Editing options 

Others \\

\textbf{15. IF YOU HAVE SELECTED OTHERS OR WISH TO PROVIDE ADDITIONAL INFORMATION, , PLEASE SPECIFY BELOW}

\textbf{16. PLEASE PROVIDE ANY ADDITIONAL COMMENT OR FEEDBACK BELOW	
}
\hfill \break
\hfill \break
\subsubsection{3. Heuristic Evaluation}

\textbf{Methodology}  

To systematically assess the usability of the Workday job application interface, we conducted a heuristic evaluation using Jakob Nielsen’s usability heuristics. Each evaluator independently reviewed the application process while documenting instances of usability violations and inefficiencies. The evaluation was conducted across different devices (desktop and mobile) to identify platform-specific issues. The results were then consolidated into key findings based on common themes.  

\textbf{Evaluation Criteria and Observations}  

\begin{itemize}
    \item \textbf{Visibility of System Status}  
    \begin{itemize}
        \item The interface often lacks clear feedback after user actions. For instance, after submitting an application, there is no immediate confirmation, leaving users uncertain about whether their application was received.
        \item The status of applications is often vague or hidden under multiple navigation layers, making it difficult for users to track their progress.
    \end{itemize}

    \item \textbf{User Control and Freedom}  
    \begin{itemize}
        \item Users lack the ability to edit their application after submission, forcing them to restart the entire process if they notice an error.
        \item Navigation between application sections is cumbersome, with no clear way to return to previous sections without losing progress.
    \end{itemize}

    \item \textbf{Consistency and Standards}  
    \begin{itemize}
        \item Different companies using Workday have varying interface layouts and application flows, leading to inconsistencies and confusion for repeat users.
        \item Icons and buttons sometimes lack uniformity in design and placement, making the interface feel disjointed.
    \end{itemize}

    \item \textbf{Match Between System and Real-World Expectations}  
    \begin{itemize}
        \item Users expect that uploading a résumé will auto-fill job history fields, but Workday often requires manual re-entry of the same information.
        \item The terminology used in error messages and instructions is sometimes unclear, requiring users to guess at corrective actions.
    \end{itemize}

    \item \textbf{Error Prevention}  
    \begin{itemize}
        \item Many users experience session timeouts without warning, resulting in lost progress and frustration.
        \item Form validation is weak, allowing users to proceed with incomplete or incorrectly formatted fields, only to encounter an error message later in the process.
    \end{itemize}
\end{itemize}

\textbf{Conclusion}  

Our heuristic evaluation revealed multiple usability flaws in Workday’s job application system, particularly in feedback mechanisms, navigation, and error handling. The findings from this evaluation will directly inform our redesign approach, prioritizing improvements in application transparency, user control, and streamlined workflows.

\newpage

\subsection{Appendix B: Needfinding Results}

\subsubsection*{Interview Summaries}  

We conducted interviews with job seekers who have used Workday for job applications. Below are summarized insights from three participants:  

\textbf{Participant 1:}  
\begin{itemize}  
    \item Applied for multiple jobs through Workday and finds it time-consuming.  
    \item Expressed frustration over entering work experience manually despite uploading a résumé.  
    \item Feels lost after applying—does not receive clear feedback on application status.  
\end{itemize}  

\textbf{Participant 2:}  
\begin{itemize}  
    \item Avoids jobs that use Workday due to past negative experiences.  
    \item Finds the mobile experience particularly frustrating.  
    \item Wishes Workday had a dashboard to track all applications in one place.  
\end{itemize}  

\textbf{Participant 3:}  
\begin{itemize}  
    \item Has applied to both internships and full-time jobs through Workday.  
    \item Notices inconsistencies between different company portals, making navigation confusing.  
    \item Often gets logged out unexpectedly, losing progress.  
\end{itemize}  
\hfill \break  

\subsubsection*{Heuristic Evaluation Notes}  

Our team conducted a heuristic evaluation of Workday’s job application system. The following usability issues were identified:   

\textbf{Visibility of System Status}  
\begin{itemize}  
    \item No clear confirmation after application submission; users often unsure if their application was received.  
    \item No estimated timeline for review or next steps.  
\end{itemize}   

\textbf{Match Between System and Real-World Expectations}  
\begin{itemize}  
    \item Users expect résumé parsing to auto-fill information but must enter details manually.  
    \item Application flow does not follow a logical sequence (e.g., references before basic details).  
\end{itemize}  


\textbf{User Control and Freedom}  
\begin{itemize}  
    \item No way to edit the application after submission.  
    \item Poor navigation makes it hard to backtrack and correct mistakes.  
\end{itemize}   

\textbf{Consistency and Standards}  
\begin{itemize}  
    \item Workday portals vary across companies, creating an inconsistent experience.  
    \item Some fields require unnecessary details, while others provide too little guidance.  
\end{itemize}  

\textbf{Error Prevention}  
\begin{itemize}  
    \item Users often time out due to inactivity, leading to lost progress.  
    \item No warning prompts before submitting an incomplete application.  
\end{itemize}   

\subsubsection*{Online Reviews Analysis}  

We analyzed over 15 online complaints from forums and review sites. The most common themes included:  

\textbf{Repetitive Data Entry}  
\begin{itemize}  
    \item Many users expressed frustration about entering résumé information multiple times.  
\end{itemize}   

\textbf{Lack of Transparency}  
\begin{itemize}  
    \item Candidates often feel their applications disappear into a "black hole" with no updates.  
\end{itemize}   

\textbf{Mobile Usability Issues}  
\begin{itemize}  
    \item Many users struggle to apply via mobile due to formatting and login issues.  
\end{itemize}  

\textbf{Inconsistent Company Portals}  
\begin{itemize}  
    \item Differences between Workday implementations across employers add to user confusion.  
\end{itemize}  


\subsection{Appendix C: Brainstorming Results - Raw Ideas}

1. AI-powered resume parser with machine learning improvements

2. Universal Workday profile that persists across employers

3. LinkedIn integration for automatic profile population

4. Template-based profile versions for different job types

5. One-click apply with minimal required fields

6. Contextual auto-completion based on past applications

7. Profile completeness indicator with targeted suggestions

8. Bulk application feature with personalization options

9. "What happens next" predictive guidance

\hfill \break
\subsection{Appendix D: Smart Profile}
 
\begin{itemize}
    \item \textbf{Intelligent Resume Parsing:} The system extracts key details from uploaded resumes, including contact information, work experience, education, and skills, using machine learning to improve accuracy over time.
    
    \item \textbf{Persistent User Profile:} Users create a single comprehensive profile that automatically populates application fields across all Workday instances, reducing redundant data entry.
    
    \item \textbf{Version Control for Job Types:} The profile supports multiple versions tailored for different job applications, allowing users to switch between profiles optimized for various industries or roles.
    
    \item \textbf{Real-time Progress and Validation:} Parsing results are displayed with confidence indicators, and inline validation highlights discrepancies, ensuring data accuracy before submission.
    
    \item \textbf{Smart Suggestions for Profile Completeness:} The system recommends missing details based on industry standards and target job requirements, helping users create a stronger application profile.
    
    \item \textbf{Privacy Manager:} Users can control which portions of their profile are visible to different employers, addressing privacy concerns while maintaining application flexibility.
    
    \item \textbf{Consistent and Intuitive UI:} A clean visual hierarchy, collapsible sections, and highlighted primary actions reduce cognitive load and improve usability, ensuring a seamless experience.
\end{itemize}

\hfill \break
\subsection{Appendix E: Interactive Application Dashboard}
\begin{itemize}
    \item \textbf{Centralized Application Tracking:} The dashboard consolidates all job applications into a single interface, presenting active applications with color-coded status indicators and deadlines for quick progress tracking.
    
    \item \textbf{Customizable Grid Layout:} Users can arrange widget-like sections such as Active Applications, Recommended Jobs, Upcoming Tasks, and Recent Activity to personalize their job search experience.
    
    \item \textbf{Intelligent Job Recommendations:} A smart matching algorithm suggests relevant job opportunities based on profile data and application history, displaying a “match percentage” with transparent explanations.
    
    \item \textbf{Proactive Task Management:} Upcoming interviews, assessments, and application deadlines are aggregated into a dedicated section, with quick access to preparation resources and calendar integration.
    
    \item \textbf{Application Analytics Panel:} Users receive insights on job search patterns, success rates, and comparative data from anonymized job seekers, helping them refine their application strategies.
    
    \item \textbf{Responsive Multi-Device Design:} The dashboard adapts to different screen sizes, transforming into a tabbed interface on mobile devices to ensure critical notifications remain accessible.
\end{itemize}

\hfill \break
\subsection{Appendix F: Smart Application Flow}
\begin{itemize}
    \item \textbf{Modular Card-Based Interface}: Breaks applications into manageable sections with collapsible cards, allowing users to navigate freely while preserving progress.
    \item \textbf{Auto-Save and Drafts}: Ensures user data is continuously saved, allowing incomplete applications to be resumed later or used as templates for future applications.
    \item \textbf{Context-Aware Pre-Filling}: Recognizes and auto-fills fields using previously submitted data, reducing redundant entry while maintaining user control.
    \item \textbf{Real-Time Validation and Feedback}: Uses dynamic validation with color-coded indicators to highlight errors and suggest improvements during form completion.
    \item \textbf{Intelligent Assistance}: Provides tailored hints and recommendations based on job descriptions to enhance application quality and relevance.
    \item \textbf{Responsive and Accessible Design}: Adapts to different screen sizes, ensuring usability on mobile and desktop with a consistent user experience.
    \item \textbf{Secure Document Locker}: Stores resumes, cover letters, and certifications with version control, enabling seamless document retrieval for applications.
\end{itemize}

\subsection{Appendix G: Evaluation Planning Materials}

\subsubsection{A.1 Participant Recruitment Screening Questionnaire}

\begin{itemize}
    \item \textbf{Demographic Information}
    \begin{itemize}
        \item Age range: 18-24, 25-34, 35-44, 45-54, 55+
        \item Gender: Male, Female, Non-binary, Prefer not to say
        \item Education level: High school, Bachelor's, Master's, PhD, Other
        \item Current employment status: Student, Employed, Unemployed, Self-employed
    \end{itemize}
    
    \item \textbf{Experience with Workday}
    \begin{itemize}
        \item Have you used Workday for job applications before? (Yes/No)
        \item If yes, when was the last time you used it? (Within last month, 1-6 months ago, 6-12 months ago, More than a year ago)
        \item How many jobs have you applied to using Workday? (1-3, 4-10, 10+)
        \item How would you rate your experience with Workday? (1-5 scale, 1=Very negative, 5=Very positive)
    \end{itemize}
    
    \item \textbf{Career Stage}
    \begin{itemize}
        \item How many years of professional experience do you have? (0-2, 3-7, 8+)
        \item How frequently do you apply for jobs? (Actively searching now, Occasionally browse, Not currently looking)
        \item What industry do you primarily work in? (Open-ended)
    \end{itemize}
\end{itemize}

\subsubsection{A.2 Consent Form Template}

\begin{center}
\textbf{PARTICIPANT CONSENT FORM}\\
Workday Job Application Interface Evaluation Study
\end{center}

\textbf{Purpose:} You are invited to participate in a research study evaluating three prototype interfaces for job applications on Workday. We aim to determine which design provides the best user experience.

\textbf{Procedures:} If you agree to participate, you will:
\begin{itemize}
    \item Complete a brief demographic questionnaire
    \item Interact with three prototype interfaces
    \item Complete two standardized tasks on each interface
    \item Provide verbal feedback during the process (think-aloud)
    \item Complete a questionnaire about your experience
    \item Participate in a brief interview about your impressions
\end{itemize}

The entire session will take approximately 45 minutes. With your permission, we will record the screen and audio during the session.

\textbf{Risks and Benefits:} There are no anticipated risks associated with this study. The benefit is contributing to the improvement of job application interfaces.

\textbf{Compensation:} You will receive a \$20 gift card for your participation.

\textbf{Confidentiality:} All data collected will be anonymized. Your personal information will not be associated with your responses. Recordings will be deleted after analysis is complete.

\textbf{Voluntary Participation:} Your participation is voluntary. You may withdraw at any time without penalty.

\textbf{Contact Information:} If you have questions about this study, please contact [Research Team Contact Information].

\vspace{0.5cm}
\rule{8cm}{0.5pt}

\textbf{Consent Statement:} I have read and understand the above information. I agree to participate in this study.

\vspace{0.5cm}
Participant Name: \rule{8cm}{0.5pt}

Participant Signature: \rule{8cm}{0.5pt}

Date: \rule{4cm}{0.5pt}

\subsubsection{A.3 Task Scenarios}

\subsubsubsection{Task 1: Create a profile and submit a job application}

Scenario: You are interested in applying for a Software Developer position at TechCorp. Using the prototype interface:

\begin{enumerate}
    \item Create or update your professional profile
    \item Search for the Software Developer position at TechCorp
    \item Review the job description
    \item Complete and submit an application for this position
\end{enumerate}

\subsubsubsection{Task 2: Track application status and respond to an interview request}

Scenario: It has been one week since you applied for the Software Developer position at TechCorp. Using the prototype interface:

\begin{enumerate}
    \item Check the status of your application
    \item View any updates or notifications
    \item Respond to an interview request for next Tuesday at 2:00 PM
    \item Add the interview to your calendar
\end{enumerate}

\subsubsection{A.4 Evaluation Metrics Collection Instruments}

\subsubsection{System Usability Scale (SUS) Questionnaire}

Rate the following statements from 1 (Strongly Disagree) to 5 (Strongly Agree):

\begin{enumerate}
    \item I think that I would like to use this system frequently.
    \item I found the system unnecessarily complex.
    \item I thought the system was easy to use.
    \item I think that I would need the support of a technical person to be able to use this system.
    \item I found the various functions in this system were well integrated.
    \item I thought there was too much inconsistency in this system.
    \item I would imagine that most people would learn to use this system very quickly.
    \item I found the system very cumbersome to use.
    \item I felt very confident using the system.
    \item I needed to learn a lot of things before I could get going with this system.
\end{enumerate}

\subsubsection{Feature-Specific Satisfaction Questionnaire}

Rate your satisfaction with the following features on a scale from 1 (Very Dissatisfied) to 7 (Very Satisfied):

\begin{enumerate}
    \item Ease of profile creation
    \item Clarity of application status
    \item Efficiency of data entry
    \item Intuitiveness of navigation
    \item Overall satisfaction
\end{enumerate}

\subsubsection{Semi-Structured Interview Guide}

\begin{enumerate}
    \item What aspects of this interface did you find most helpful for the job application process?
    \item What aspects were confusing or difficult to use?
    \item How does this interface compare to your previous experiences with job application systems?
    \item What feature would you most want to see improved or added?
    \item If you were to describe this interface to a friend, what would you say?
    \item Do you have any other thoughts or feedback you'd like to share about this interface?
\end{enumerate}

\subsubsection{Observer Coding Sheet}

\begin{tabular}{|p{3cm}|p{3cm}|p{3cm}|p{3cm}|}
\hline
\textbf{Observed Behavior} & \textbf{Prototype 1 (Smart Profile)} & \textbf{Prototype 2 (Interactive Dashboard)} & \textbf{Prototype 3 (Smart Application Flow)} \\
\hline
Hesitations & & & \\
\hline
Errors & & & \\
\hline
Backtracking & & & \\
\hline
Expressions of Frustration & & & \\
\hline
Expressions of Success & & & \\
\hline
Questions Asked & & & \\
\hline
Notable Quotes & & & \\
\hline
\end{tabular}

\hfill \break

\subsection{Appendix H: Evaluation Results}
\subsubsection{B.1 Participant Demographics Details}
\subsubsection{B.2 Detailed Task Completion Times}


\begin{table}[h]
\centering
\begin{tabular}{|l|c|}
\hline
\textbf{Characteristic} & \textbf{Distribution} \\
\hline
Age Range & 22-29 (7), 30-39 (5), 40-45 (3) \\
\hline
Gender & Female (9), Male (6) \\
\hline
Education & Bachelor's (8), Master's (6), PhD (1) \\
\hline
Employment Status & Employed (8), Unemployed (4), Student (3) \\
\hline
Industry Experience & Tech (6), Healthcare (3), Finance (2), Education (2), Other (2) \\
\hline
Workday Experience & 1-3 applications (5), 4-10 applications (7), 10+ applications (3) \\
\hline
\end{tabular}
\caption{Detailed participant demographics}
\end{table}



\begin{table}[h]
\centering
\begin{tabular}{|c|c|c|c|c|c|c|}
\hline
\multirow{2}{*}{\textbf{Participant}} & \multicolumn{2}{c|}{\textbf{Smart Profile}} & \multicolumn{2}{c|}{\textbf{Interactive Dashboard}} & \multicolumn{2}{c|}{\textbf{Smart Application Flow}} \\
\cline{2-7}
 & \textbf{Task 1} & \textbf{Task 2} & \textbf{Task 1} & \textbf{Task 2} & \textbf{Task 1} & \textbf{Task 2} \\
\hline
P1 & 118 & 68 & 156 & 52 & 98 & 61 \\
\hline
P2 & 142 & 85 & 165 & 63 & 120 & 70 \\
\hline
P3 & 137 & 74 & 128 & 49 & 102 & 57 \\
\hline
P4 & 103 & 64 & 138 & 55 & 95 & 62 \\
\hline
P5 & 128 & 82 & 147 & 72 & 110 & 68 \\
\hline
P6 & 115 & 71 & 125 & 60 & 91 & 58 \\
\hline
P7 & 132 & 89 & 152 & 65 & 119 & 75 \\
\hline
P8 & 120 & 77 & 136 & 58 & 103 & 66 \\
\hline
P9 & 145 & 85 & 167 & 70 & 125 & 69 \\
\hline
P10 & 110 & 73 & 124 & 51 & 89 & 54 \\
\hline
P11 & 129 & 68 & 145 & 49 & 104 & 61 \\
\hline
P12 & 133 & 79 & 150 & 62 & 115 & 67 \\
\hline
P13 & 122 & 84 & 139 & 57 & 99 & 65 \\
\hline
P14 & 119 & 72 & 130 & 54 & 104 & 60 \\
\hline
P15 & 117 & 69 & 133 & 62 & 101 & 63 \\
\hline
\textbf{Mean} & 124.3 & 76.2 & 142.1 & 58.5 & 105.8 & 63.7 \\
\hline
\textbf{SD} & 18.5 & 12.4 & 22.7 & 9.6 & 15.3 & 10.8 \\
\hline
\end{tabular}
\caption{Individual task completion times in seconds for each participant}
\end{table}

\subsubsection{B.3 Full SUS Scores}
\begin{table}[h]
\centering
\begin{tabular}{|c|c|c|c|}
\hline
\textbf{Participant} & \textbf{Smart Profile} & \textbf{Interactive Dashboard} & \textbf{Smart Application Flow} \\
\hline
P1 & 75 & 72.5 & 85 \\
\hline
P2 & 70 & 75 & 80 \\
\hline
P3 & 82.5 & 85 & 90 \\
\hline
P4 & 77.5 & 80 & 87.5 \\
\hline
P5 & 65 & 67.5 & 75 \\
\hline
P6 & 80 & 82.5 & 90 \\
\hline
P7 & 72.5 & 77.5 & 82.5 \\
\hline
P8 & 75 & 72.5 & 85 \\
\hline
P9 & 62.5 & 65 & 72.5 \\
\hline
P10 & 85 & 87.5 & 92.5 \\
\hline
P11 & 77.5 & 82.5 & 87.5 \\
\hline
P12 & 67.5 & 72.5 & 77.5 \\
\hline
P13 & 75 & 77.5 & 85 \\
\hline
P14 & 80 & 85 & 87.5 \\
\hline
P15 & 72.5 & 70 & 80 \\
\hline
\textbf{Mean} & 74.2 & 76.8 & 83.5 \\
\hline
\textbf{SD} & 8.7 & 9.2 & 7.4 \\
\hline
\end{tabular}
\caption{Individual System Usability Scale scores for each participant}
\end{table}

\subsubsection{B.4 Complete Feature-Specific Satisfaction Ratings}

\begin{table}[h]
\centering
\begin{tabular}{|l|c|c|c|c|c|}
\hline
\multirow{2}{*}{\textbf{Prototype}} & \multicolumn{5}{c|}{\textbf{Feature Rating (Mean)}} \\
\cline{2-6}
 & \textbf{Profile Creation} & \textbf{Status Clarity} & \textbf{Data Entry} & \textbf{Navigation} & \textbf{Overall} \\
\hline
Smart Profile & 5.3 & 4.5 & 5.8 & 4.9 & 5.2 \\
\hline
Interactive Dashboard & 4.7 & 6.4 & 4.3 & 5.6 & 5.5 \\
\hline
Smart Application Flow & 6.1 & 5.2 & 6.2 & 5.9 & 6.3 \\
\hline
\end{tabular}
\caption{Mean feature-specific satisfaction ratings on 7-point Likert scale}
\end{table}

\begin{table}[h]
\centering
\begin{tabular}{|c|c|c|c|c|c|c|c|c|c|c|c|c|c|c|c|}
\hline
\multirow{2}{*}{\textbf{Prototype}} & \multicolumn{15}{c|}{\textbf{Overall Satisfaction Rating by Participant}} \\
\cline{2-16}
 & \textbf{P1} & \textbf{P2} & \textbf{P3} & \textbf{P4} & \textbf{P5} & \textbf{P6} & \textbf{P7} & \textbf{P8} & \textbf{P9} & \textbf{P10} & \textbf{P11} & \textbf{P12} & \textbf{P13} & \textbf{P14} & \textbf{P15} \\
\hline
Smart Profile & 6 & 5 & 6 & 5 & 4 & 6 & 5 & 5 & 4 & 6 & 5 & 4 & 5 & 6 & 5 \\
\hline
Interactive Dashboard & 5 & 6 & 6 & 5 & 4 & 6 & 5 & 5 & 4 & 7 & 6 & 5 & 6 & 7 & 5 \\
\hline
Smart Application Flow & 7 & 6 & 7 & 6 & 5 & 7 & 6 & 6 & 5 & 7 & 7 & 6 & 7 & 7 & 6 \\
\hline
\end{tabular}
\caption{Individual overall satisfaction ratings on 7-point Likert scale}
\end{table}

\subsubsection{B.5 Thematic Analysis Coding Framework}

\begin{table}[h]
\centering
\begin{tabular}{|l|p{10cm}|}
\hline
\textbf{Theme Category} & \textbf{Description and Sub-themes} \\
\hline
Efficiency & Time savings, reduced effort, automation features, minimizing repetitive tasks \\
\hline
Clarity & Clear instructions, transparent process, understanding status, predictability \\
\hline
Control & User autonomy, customization options, decision-making power, flexibility \\
\hline
Emotional Response & Stress reduction, confidence in system, satisfaction, frustration reduction \\
\hline
Learnability & Intuitiveness, familiarity, clear patterns, ease of understanding \\
\hline
Visual Design & Layout, colors, typography, visual hierarchy, information density \\
\hline
Informational Content & Helpfulness of information, relevance, accuracy, completeness \\
\hline
Technical Performance & Speed, reliability, error handling, responsiveness \\
\hline
\end{tabular}
\caption{Qualitative analysis coding framework}
\end{table}

\subsubsection{B.6 Selected Participant Quotes}

\subsubsection{Smart Profile}

\begin{quote}
"The automated resume parsing was impressive—it saved me at least 10 minutes of tedious data entry." (P3)
\end{quote}

\begin{quote}
"I really liked being able to maintain different versions of my profile for different job types, but I was confused about how to effectively use the confidence indicators." (P8)
\end{quote}

\begin{quote}
"The privacy controls felt overwhelming. I spent too much time configuring settings before I could even start applying." (P12)
\end{quote}

\subsubsection{Interactive Dashboard}

\begin{quote}
"The status visualization was excellent—I could see exactly where each application was in the process without having to click through multiple screens." (P6)
\end{quote}

\begin{quote}
"I found the match percentages for recommended jobs really helpful for prioritizing which positions to look at first." (P10)
\end{quote}

\begin{quote}
"There's just too much happening on one screen. It looks powerful, but it's intimidating when you first see it." (P5)
\end{quote}

\subsubsection{Smart Application Flow}

\begin{quote}
"Breaking everything into cards made the whole process feel manageable. I didn't have that usual dread of facing a giant form." (P4)
\end{quote}

\begin{quote}
"The auto-save feature was a lifesaver. I'm always worried about losing my work halfway through an application." (P7)
\end{quote}

\begin{quote}
"The suggestions for how to phrase my experience were surprisingly helpful. It's like having a career coach right in the application." (P14)
\end{quote}

\begin{quote}
"I wasn't always sure which cards were mandatory and which were optional. I needed clearer indicators." (P9)
\end{quote}

\newpage
\subsection{Appendix I: Final Prototype Smart Application Flow 2.0 Technical Specifications}
\subsubsection{I.1 Smart Application Flow 2.0 Technical Specifications}
The Smart Application Flow 2.0 prototype was developed using the following technical specifications:

\begin{itemize}
    \item \textbf{Framework:} React.js for front-end development with responsive design principles
    \item \textbf{Color Palette:}
    \begin{itemize}
        \item Primary: \#1976d2 (Blue)
        \item Secondary: \#2c3e50 (Dark Navy)
        \item Accent: \#ff8f00 (Orange)
        \item Background: \#ffffff (White)
        \item Card Background: \#f8f9fa (Light Gray)
    \end{itemize}
    \item \textbf{Typography:}
    \begin{itemize}
        \item Primary Font: Arial (Sans-serif)
        \item Header Size: 16-20px
        \item Body Text: 14px
        \item Supporting Text: 12px
    \end{itemize}
    \item \textbf{Accessibility:} All text elements maintain a minimum contrast ratio of 4.5:1 following WCAG 2.1 AA standards
\end{itemize}

\subsubsection*{I.2 Implementation Details}

\subsubsubsection*{I.2.1 Smart Profile System}
The profile system utilizes machine learning to parse resume content and intelligently populate relevant fields across the application. The system implements:

\begin{itemize}
    \item Document parsing using natural language processing
    \item Entity recognition for education, work experience, and skills
    \item Confidence scoring to highlight potentially inaccurate auto-filled fields
    \item User correction capabilities with machine learning feedback loop
\end{itemize}

\subsubsubsection*{I.2.2 Card-Based Interface Architecture}
The card-based interface implements:

\begin{itemize}
    \item Modular React components for each application section
    \item Conditional rendering based on application requirements
    \item Visual state indicators (complete, in progress, required, optional)
    \item Progressive disclosure of information to reduce cognitive load
    \item Semantic HTML5 structure for improved accessibility
\end{itemize}

\subsubsubsection*{I.2.3 Document Locker Implementation}
The document locker features:

\begin{itemize}
    \item Secure document storage with encryption
    \item Thumbnail generation for visual reference
    \item Metadata extraction (file type, date modified, size)
    \item Version control for multiple iterations of the same document
    \item Drag-and-drop interface for easy document management
\end{itemize}

\subsubsection*{I.3 Design Rationale}

\subsubsubsection*{I.3.1 Addressing Core User Frustrations}
Our design choices specifically targeted the pain points identified in our needfinding:

\begin{itemize}
    \item \textbf{Redundant Data Entry:} Solved through smart profile system with cross-application data persistence
    \item \textbf{Unclear Requirements:} Addressed with visual indicators for required fields and contextual help
    \item \textbf{Complex Navigation:} Simplified through card-based modular approach with clear progress tracking
    \item \textbf{Document Management:} Enhanced with centralized document locker with visual previews
\end{itemize}

\subsubsubsection*{I.3.2 Application of Design Principles}
The prototype applies several key design principles:

\begin{itemize}
    \item \textbf{Progressive Disclosure:} Information is presented in manageable chunks
    \item \textbf{Recognition over Recall:} Visual elements help users recognize options rather than recall them
    \item \textbf{Error Prevention:} Smart suggestions and validation help prevent common errors
    \item \textbf{Consistency:} Uniform visual language and interaction patterns throughout the application
    \item \textbf{Feedback:} Clear visual and textual feedback for user actions
\end{itemize}

\newpage


\subsection{Appendix J: Final Evaluation Planning}

\subsubsection{ Participant Task Instructions}

\subsubsubsection*{Task 1: Resume Upload and Profile Review}
\begin{enumerate}
    \item Navigate to the Smart Application Flow
    \item Upload your resume using the document locker
    \item Review how the system has populated your profile
    \item Make at least one edit to the auto-filled information
\end{enumerate}

\subsubsubsection*{Task 2: Skills Section Completion}
\begin{enumerate}
    \item Navigate to the Skills \& Qualifications section
    \item Select at least three skills that apply to you
    \item Review the smart suggestion provided
    \item Add one additional skill based on the suggestion
\end{enumerate}

\subsubsubsection*{Task 3: Navigation and Progress Testing}
\begin{enumerate}
    \item Navigate from the Skills section to the Document Locker
    \item Add a new document
    \item Return to a previous section
    \item Proceed to the next section
    \item Verify that your progress has been maintained
\end{enumerate}

\subsubsection*{Quantitative Survey Questions}

\subsubsubsection*{System Usability Scale (SUS)}
Standard 10-item SUS questionnaire with 5-point agreement scale:

\begin{enumerate}
    \item I think that I would like to use this system frequently.
    \item I found the system unnecessarily complex.
    \item I thought the system was easy to use.
    \item I think that I would need the support of a technical person to be able to use this system.
    \item I found the various functions in this system were well integrated.
    \item I thought there was too much inconsistency in this system.
    \item I would imagine that most people would learn to use this system very quickly.
    \item I found the system very cumbersome to use.
    \item I felt very confident using the system.
    \item I needed to learn a lot of things before I could get going with this system.
\end{enumerate}

\subsubsubsection*{Feature-Specific Ratings}
Rate your agreement with the following statements (1=Strongly Disagree, 7=Strongly Agree):
\begin{enumerate}
    \item The distinction between required and optional fields was clear
    \item The document management system was intuitive to use
    \item The smart suggestions provided helpful guidance
    \item The card-based interface made the application process less overwhelming
    \item The progress tracking feature helped me understand where I was in the process
    \item I could easily navigate between different sections of the application
    \item The auto-save feature gave me confidence that my work wouldn't be lost
    \item Overall, I would prefer this interface to traditional job application systems
\end{enumerate}

\subsubsubsection*{Task Performance Metrics}
\begin{itemize}
    \item Time to complete each task (recorded by facilitator)
    \item Number of errors or instances of confusion per task
    \item Binary task completion success (yes/no)
\end{itemize}

\subsubsection*{Qualitative Interview Guide}

\begin{enumerate}
    \item What aspects of the interface did you find most helpful?
    \item Were there any parts of the application process that confused you?
    \item How does this experience compare to other job application systems you've used?
    \item What additional features would make this system more useful to you?
    \item Did the smart suggestions feel helpful or intrusive?
    \item How did you feel about the balance between automated features and manual control?
    \item Is there anything else you'd like to share about your experience with this prototype?
\end{enumerate}

\subsubsection*{Data Analysis Protocol}

\subsubsubsection*{Quantitative Analysis Protocol}
\begin{itemize}
    \item Calculate mean, median, and standard deviation for SUS scores
    \item Generate descriptive statistics for all Likert-scale responses
    \item Measure task completion times and calculate averages per task
    \item Record error frequency and types across all tasks
    \item Compare results against benchmark metrics:
    \begin{itemize}
        \item SUS score target: 80+
        \item Task completion rate target: 90\%
        \item User satisfaction target: 5.5+ on 7-point scale
    \end{itemize}
\end{itemize}

\subsubsubsection*{Qualitative Analysis Protocol}
\begin{itemize}
    \item Transcribe all interview responses
    \item Conduct open coding to identify initial themes
    \item Group similar codes into categories
    \item Identify recurring patterns across participants
    \item Map qualitative insights to quantitative results
    \item Extract actionable design recommendations
\end{itemize}

\subsubsection*{Participant Recruitment Materials}

\subsubsubsection*{Sample Recruitment Message}
\begin{quote}
Seeking participants for a 15-minute user study on job application interfaces! We are evaluating a new prototype designed to make applying for jobs easier and more efficient. Requirements: must have applied for at least one job online in the past year. To participate, please complete the screening form at [link].
\end{quote}

\subsubsubsection*{Screening Criteria}
\begin{itemize}
    \item Has applied to at least one job online in the past year
    \item Represents diverse demographic backgrounds
    \item Mix of undergraduate and graduate students
    \item Varied levels of job application experience
    \item Different academic majors and career interests
\end{itemize}

\subsubsection*{Consent Form}

\begin{quote}
\textbf{Study Title:} Evaluation of Smart Application Flow Interface\\
\textbf{Purpose:} To evaluate the usability and effectiveness of a new job application interface

\textbf{Procedures:}
\begin{itemize}
    \item You will interact with a prototype job application interface
    \item You will complete a short survey and answer interview questions
    \item The session will last approximately 15 minutes
    \item Screen and audio will be recorded for analysis purposes only
\end{itemize}

\textbf{Risks \& Benefits:}
\begin{itemize}
    \item There are minimal risks associated with this study
    \item Benefits include contributing to improved job application systems
    \item You will receive a \$10 gift card as compensation
\end{itemize}

\textbf{Confidentiality:}
\begin{itemize}
    \item All data will be anonymized
    \item Recordings will be deleted after analysis
    \item Results will be presented in aggregate form
\end{itemize}

\textbf{Voluntary Participation:} Your participation is voluntary, and you may withdraw at any time.

I have read and understood the above information and agree to participate in this study.
\end{quote}

\newpage

\subsection{Appendix K: Final Evaluation Results}
\subsubsection{K.1 Participant Demographics}

We recruited a total of 15 participants to evaluate our Smart Application Flow 2.0 prototype. All participants had prior experience using Workday, with varying levels of familiarity and application frequency.

\begin{itemize}
    \item \textbf{Education Level:} 
    \begin{itemize}
        \item 9 Graduate Students (Master’s level)
        \item 6 Undergraduate Students (3rd and 4th year)
    \end{itemize}

    \item \textbf{Experience with Job Applications:}
    \begin{itemize}
        \item 6 actively applying for internships
        \item 5 actively applying for full-time positions
        \item 4 casually browsing for jobs or planning future applications
    \end{itemize}

    \item \textbf{Workday Usage:}
    \begin{itemize}
        \item 12 participants used Workday within the last 6 months
        \item 3 had prior experience more than 1 year ago
    \end{itemize}

    \item \textbf{Device Usage During Evaluation:}
    \begin{itemize}
        \item 8 participants used laptops/desktops
        \item 7 participants used mobile phones or tablets
    \end{itemize}

    \item \textbf{Age Range:}
    \begin{itemize}
        \item 18–24 years: 10 participants
        \item 25–34 years: 5 participants
    \end{itemize}
\end{itemize}

\subsubsection{K.2 Task Completion Metrics}

Participants were asked to complete three tasks using the prototype. Below are the performance metrics across all participants:

\begin{table}[h]
\centering
\begin{tabular}{|l|c|c|c|}
\hline
\textbf{Task} & \textbf{Average Time (sec)} & \textbf{Avg. Errors} & \textbf{Completion Rate} \\
\hline
Resume Upload \& Review Auto-Fill (Task 1) & 105.2 & 0.6 & 100\% \\
Complete Skills Section (Task 2) & 85.4 & 0.2 & 100\% \\
Navigate Between Sections (Task 3) & 70.3 & 0.3 & 100\% \\
\hline
\end{tabular}
\caption{Task-based performance metrics}
\end{table}

\textbf{Key Observations:}
\begin{itemize}
    \item 4 participants hesitated after auto-filling, unsure which fields were editable
    \item The document locker was overlooked by 3 participants initially, despite being visible
    \item All participants completed tasks without needing assistance or external guidance
\end{itemize}

\subsubsection{K.3 System Usability Scale (SUS)}

\begin{itemize}
    \item \textbf{Mean SUS Score:} 85.0
    \item \textbf{Standard Deviation:} 5.9
    \item \textbf{SUS Score Range:} 75 to 92
\end{itemize}

According to industry standards, a score above 80.3 is considered “excellent.” 12 out of 15 participants rated the prototype above 80, indicating strong usability and overall satisfaction.

\subsubsection{K.4 Feature-Specific Satisfaction Ratings}

Participants rated several key features on a 7-point Likert scale:

\begin{table}[h]
\centering
\begin{tabular}{|l|c|c|}
\hline
\textbf{Feature} & \textbf{Mean Score} & \textbf{Standard Deviation} \\
\hline
Clarity of required vs. optional fields & 6.0 & 0.7 \\
Ease of document management & 5.8 & 0.9 \\
Helpfulness of smart suggestions & 6.1 & 0.6 \\
Visual layout / spacing & 6.3 & 0.5 \\
Overall satisfaction & 6.4 & 0.5 \\
\hline
\end{tabular}
\caption{Participant satisfaction with specific features}
\end{table}

\subsubsection{K.5 Qualitative Insights}

Semi-structured post-session interviews yielded several key themes:

\textbf{Positively Perceived Features}
\begin{itemize}
    \item \textbf{Modular Flow:} “The card-style navigation made it less stressful—more bite-sized and less overwhelming.”
    \item \textbf{Auto-Save Indicator:} “Seeing the ‘last saved’ message gave me peace of mind, especially compared to Workday’s sudden timeouts.”
    \item \textbf{Resume Parsing Accuracy:} “It got 90\% of my resume right—way better than what I expected.”
    \item \textbf{Navigation Flow:} “The top breadcrumb trail made it easy to understand where I was at all times.”
\end{itemize}

\textbf{Pain Points and Suggestions}
\begin{itemize}
    \item \textbf{Required vs. Optional Fields:} “I wasn't always sure what I needed to fill in, some stronger labels would help.”
    \item \textbf{Discoverability of Document Locker:} “Took me a while to find the document locker. Maybe highlight it on first use?”
    \item \textbf{Advanced Users:} “Would love a ‘condensed view’ or keyboard shortcuts for fast applications.”
    \item \textbf{Onboarding Prompt:} “Some kind of tooltip walkthrough on first load would be helpful.”
\end{itemize}

\textbf{General Sentiment}
\begin{itemize}
    \item “Honestly, this feels like what Workday should already be.”
    \item “I would absolutely use this if it were available today.”
    \item “Clean, non-intimidating, and efficient, that’s rare for job sites.”
\end{itemize}

\subsubsection{K.6 Summary and Reflection}

The data strongly supports the usability of our Smart Application Flow 2.0 prototype. Quantitative metrics showed high completion rates and satisfaction, while qualitative feedback validated the core design improvements. Remaining suggestions focused on minor clarity and visibility issues, which can be addressed in future iterations. The overall response demonstrated that our redesign meaningfully improves the user experience for job seekers compared to the traditional Workday interface.


\end{sloppypar}
\end{document}
