\documentclass[
	%a4paper, % Use A4 paper size
	letterpaper, % Use US letter paper size
]{jdf}

\addbibresource{references.bib}

\author{Omair Tariq Khan}
\email{okhan60@gatech.edu}
\title{Homework 1}

\begin{document}
%\lsstyle

\maketitle
\hfill \break
\hfill \break


\subsection{Answer to Question 1 - Selecting and Using Apple Music}
I use Apple Music in a variety of ways, and I've noticed that each environment presents its own set of challenges, which significantly affect how I interact with the app. Whether I’m walking, sitting at home, or exercising, the user experience with the app shifts depending on my surroundings. When I’m walking down the street or navigating through a busy area, it’s harder to focus on the app because of distractions like traffic and people. It’s also tricky to tap small buttons while moving. In this context, the app could simplify the experience to reduce the extra "precision" that's needed. For example, physical buttons for things like play/pause could be introduced like the more accessible buttons for volume control that exist already. 

It would also be great if Apple integrated voice commands more prominently, allowing me to control the Music app hands-free while staying focused on the environment around me, especially in the case of non-English titles. In contrast, when I’m sitting at home or in a quiet space, I have more time and mental space to engage with Music. In this scenario, I tend to explore new albums and create playlists. The interface can afford to be more complex, letting me browse my music collection and discover new songs without any pressure or distractions. 

However, when I’m exercising, whether at the gym or outdoors the situation changes again. I’m often sweaty or wearing gloves, so this constant movement makes it hard to use small touch controls. I may also have limited hand dexterity during a workout. The action button doesn't have the capability to forward or rewind your playlist, although I think it is a handy button because it can be touched without looking at the phone itself. I also think that Apple Music could benefit from a more fitness-friendly interface. Integrating with fitness trackers could allow me to tailor the music experience based on my workout intensity, keeping me motivated and focused without needing to manually adjust the app.

To improve the experience even further, I think Apple Music could be designed to adjust dynamically depending on my context. For example, when I’m walking or driving, the app could simplify the interface and highlight larger controls. I am not looking at the album art while walking, so if there was a detection system in place, it would display the larger multimedia controls on my screen. Newer phones also have the capability of an always on screen. Multimedia buttons can also be added to that. As a user, I'm more prone to be connected to Bluetooth in my earphones or my Car or Apple CarPlay when I'm not home. To add to my point from above, Apple Music has forward, backward, and pause buttons feature on Google Maps but not Apple Maps, which does highlight its ineffectiveness. At home, the app could offer more detailed options for music exploration, such as more intricate playlist management and album browsing. 

In conclusion, using Apple Music in different situations comes with its own challenges. If it detects me walking or driving automatically, that would be great. I feel like with Siri handling basic commands like opening and closing the app itself, there just needs to be a bit more intuition on the app automatically adjusting based on these different environments. This would provide a more seamless and enjoyable experience, no matter where I am or what I’m doing. 
\newpage

\subsection{Answer to Question 2}
The process of submitting a question and receiving an answer on Ed Discussion involves several steps that highlight how the system supports and/or challenges students in achieving their goals. These steps can be analyzed using the concepts of the gulf of execution and the gulf of evaluation, focusing on how the platform helps bridge these gaps.

\subsubsection {the Gulf of Execution}
The gulf of execution begins with the student forming a goal, such as asking for clarification on a concept or receiving help with an assignment. Ed Discussion supports this planning stage by providing clear entry points for creating a new post. The “New Thread” button is prominently displayed on the interface, and categories such as “Class Discussion” or “Homework” help students know where to post. It’s also nice that you can tag your thread as either a "Post" or a "Question", each with the small icons, e.g. a "message" or "question mark", which is great. 

After deciding to ask a question, the student needs to figure out how to write it. Ed Discussion helps by giving a form with labeled boxes for the title, the question, and the category. It also has dropdown menus to make it simple to choose options. Although some students still might struggle with understanding how much detail to include or which category fits best. But, this is addressed by showing questions already asked with similar keywords that guide students in framing their questions appropriately. Making an example (rough) thread just to get an idea of the layout is good, just like we were allowed to for this Homework!

Finally, students execute their plan by filling out the form, selecting the category, and submitting their post/question. Ed Discussion provides immediate visual feedback by displaying the question on the discussion board, but it is not tagged as “Unanswered.” This does reassure students that their question has been successfully posted. However, students have to filter by "Unanswered" from the drop-down to see whether their question is answered or not. 

\subsubsection {The Gulf of Evaluation}
The gulf of evaluation begins once the question is submitted, and students need to understand the outcome of their actions. Ed Discussion helps by showing the question under the right category and has "views" and "Watching" buttons there. This visual feedback ensures that students can see their question and verify its presence on the platform. However, if the interface were less responsive or the feedback were delayed, students might worry whether their question was posted successfully.

In the interpretation stage, students must understand the status of their question and what happens next. Ed Discussion supports this by making the question visible to peers, TAs, and professors, who can then reply. The “Unanswered” tag from the drop down signals that the question is pending a response, but if there are tons of questions posted, your post can get lost. Though the filter by "Me" button does help a bit, however, some students may be unsure of the expected response time or whether their question was clear. Adding features like an estimated response time (calculating from average response time over the past few weeks, maybe?) or a confirmation that the question was received and is being reviewed could enhance this stage.

The comparison stage involves evaluating the answer after it has been received. When a TA, professor, or peer responds, Ed Discussion notifies the student and displays the response below the original question in a threaded format. This organization may or may not help students understand how the response addresses their query because lines that branch down from a response are sometimes harder to follow as to who is responding to who. If notifications are delayed or hard to see, students might miss the response. To address this, the platform does implement personalized notifications, such as email alerts.

Another way to improve the system would be to add features like small tips when hovering over buttons, or showing if someone is typing a response. Varying feedback methods, such as visual indicators showing the number of likes or people typing and auditory cues for new responses, could keep students informed and engaged. I would also like to see an option of a "personal" thread where I can post notes, or thoughts without anyone seeing them as right now, you can only hide your threads from students and not the course instructor/staff. 
\newpage

\subsection{Answer to Question 3}
\subsubsection {Activity 1: Using a Self-Checkout Machine at a Grocery Store}
Whenever I use a self-checkout machine at the grocery store, I often find it hard to figure out how to use it properly. One of the main issues is scanning produce. The machine isn’t very user-friendly, and I spend a lot of time searching through long lists of items just to find something basic like bananas. The options sometimes aren’t clear, and I feel like I’m guessing half the time. If I make a mistake, like accidentally scanning an item twice, removing it from the cart feels overly complicated because now you have to wait for someone to get to you and scan their badge. 
Sometimes, even basic tasks like paying can be confusing. I’ve had moments where the machine didn’t register that I inserted my card, leaving me staring at the screen wondering if I did something wrong or if the machine just froze. The lack of clear instructions makes the whole process stressful. 

The gulf of evaluation is also a problem with these machines. For example, when I scan an item, the machine doesn’t always respond right away, and I’ve accidentally scanned the same item twice because I wasn’t sure it went through the first time.  Then there’s the dreaded “Unexpected item in the bagging area” error. It’s such a vague message, and I’m never sure if I need to move the item, rescan it, or just wait for help. These unclear prompts make an already stressful process even more frustrating, especially when there’s a line of people waiting behind me. 

\subsubsection {Activity 2: Using a Ride-Sharing App (e.g., Uber)}
In contrast, I’ve had a much smoother experience using apps like Uber or even Lyft! These apps are simple to use, and they do a great job guiding me through the process. When I open the app, I can easily input my destination, and it even suggests common places based on my previous trips, which saves me time. After confirming my pickup location, the app keeps me informed every step of the way. I can see the driver’s exact location on the map, get an estimated time of arrival, and their live location. I travel frequently between my home state in New Jersey and Arizona (where I live), and I can set the exact terminal, door, coordinate position (north/south if your airport has those!!)to go home from the airport. When I am going to the Airport, I can book a ride well in advance and the dropoff point allows me to select the terminal and the door based on the airline I am traveling in (the door is many times not clear in your travel itinerary).

If there’s an issue, like a driver canceling, the app tells me exactly what happened and automatically starts looking for a new driver. I never feel lost or unsure of what’s going on because the app is constantly keeping me in the loop. Even the payment process is seamlessly done through ApplePay or your credit card, and I get a receipt right after the ride ends. The whole experience is smooth, clear, and stress-free. Also, while traveling you can share your live location with your friends/family just for your safety. 

Self-checkout machines could learn a lot from ride-sharing apps by improving both discoverability and feedback. For example, the interface could be made more user-friendly, with clear categories for produce and animations to show how to do things step by step. Real-time feedback, like a progress bar for scanning items or instructions that update as you go, would make it easier to understand what’s happening. A better error system, like a pop-up explaining what went wrong and how to fix it, would make the experience less frustrating and help build user confidence.
\newpage

\subsection{Answer to Question 4}
The use of tools like ChatGPT for therapy-like support raises important ethical questions. On one hand, these tools provide quick and easy help for people who can’t access traditional mental health care due to cost, stigma, or availability. On the other hand, ChatGPT isn’t a licensed therapist. It can’t truly understand emotions, give detailed care, or handle emergencies. This could lead to people relying on it too much and not getting the professional help when they need it most (Columbia Psychiatry, 2023).

To ethically permit such usage, OpenAI should clearly state that ChatGPT isn’t a replacement for therapy. Every interaction involving mental health topics should include disclaimers reminding users that it’s not a substitute for therapy and encouraging them to seek professional help. For example, if someone writes about self-harm or extreme distress, the tool should automatically direct them to a hotline or crisis resource right away. While these tools can’t replace real human care, they can still help people who might not have any other support. while ChatGPT can simulate empathy, it cannot replicate the nuanced human connection that is vital in therapeutic settings (Health.com, 2023). Nevertheless, the benefits of providing conversational support, especially for those who don't have immediate access to other resources could outweigh the risks if clear boundaries are established.

Proposed Experiment and Ethical Recruitment

To improve ChatGPT for mental health conversations, I suggest testing a feature that recognizes user emotions through their messages. To test this feature, an experiment would involve consenting participants using an enhanced version of ChatGPT for mental health-related conversations. Based on the emotion, the tool could suggest helpful exercises like mindfulness or ways to reframe negative thoughts. For this test, we could recruit volunteers through mental health forums and social media.

The recruitment message might say: "We’re studying how AI tools like ChatGPT can support mental well-being. This experimental version offers guided conversations and reflective exercises, but it’s not a replacement for professional care. If you’re interested, click here to learn more." Participants would sign a consent form explaining the study’s purpose, the risks, and that it’s voluntary. The consent form would also outline the available crisis resources, ensuring informed consent. Now, this approach may seem transparent, it might attract people who already trust AI tools in this regard. Though I do think that the experiment could give us valuable insights into how ChatGPT can safely support mental health. Ensuring responsible utilization of ChatGPT in psychological contexts requires a combination of caution, human judgment, and robust regulatory frameworks.


\newpage
\subsection{Tables}
You have freedom to format tables in the way that works best for your data. Generally, text should be left-aligned and numbers should be right-aligned or aligned at the decimal – you can do this using a \href{https://practicaltypography.com/tabs-and-tab-stops.html}{custom tab stop}. The default table style (shown below) reduces the text size to be equal to the caption text.

Table captions should be formatted the same way as figure captions, but they should be placed above the table. The popular mnemonic for this is: figures at the foot, tables at the top. Like figures, tables should not exceed the margins and should be centered on the page.

\begin{jdftable}
\captionof{table}{Mathematical constants. Notice how the approximations align at the decimal.}\label{table:Example}
\small % Reduce font size
\begin{tabular}{@{} L{0.15\linewidth} c S L{0.52\linewidth}}
	\textbf{Name} & \textbf{Symbol} & \textbf{Approximation} & \textbf{Description} \\
	\toprule[0.5pt]
	Golden ratio & $\phi$ & 1.618 & Number such that the ratio of " to the number is equal to the ratio of its reciprocal to 1\\
	\midrule
	Euler's number & $e$ & 2.71828 & Exponential growth constant\\
	\midrule
	Archimedes' constant & $\pi$ & 3.14 & The ratio between circumference and diameter of a circle\\
	\midrule
	One hundred & A+ & 100.00 & The grade we hope you’ll all earn in this class\\
\end{tabular}
\end{jdftable}

\subsection{Additional elements}
There are additional elements you may want to include in your paper, such as in-line or block quotes, lists, and more. For other content types not covered here, you have flexibility in determining how it should be used in this format.

\subsubsection{Quotes}
If you would like to quote an outside source, you may do so with quotation marks followed by a citation. If a quote is fewer than three lines, you may write it in-line. It is acceptable to replace pronouns with their target in brackets for clarity. For example, "Heavy use of peer grading would compromise [the school’s] reputation" \citep{joyner2016}. If a quote exceeds three lines, you should set it as its own paragraph with 0.5" side margins, using the \emph{Blockquote} style.

\begin{quotation}
"Whether or not the grades generated by peers are reliably similar to grades generated by experts is only one factor worth considering, however. Student perception is also an important factor. A recent study indicated that reliance on peer grading is one of the top drivers of high MOOC dropout rates. This problem may be addressed by reintroducing some expert grading where possible." \citep{joyner2016}
\end{quotation}

\subsubsection{Lists}
Bulleted and numbered lists are indented 0.5" from the left margin, with the bullet or number hanging in the margin by 0.25" (the default format).

Bullet points:

\begin{itemize}
	\item First bullet point item
	\item Second bullet point item
\end{itemize}

Numbered list:

\begin{enumerate}
	\item First numbered item
	\item Second numbered item
\end{enumerate}

\section{Procedural elements}
\subsection{In-line citations}
Articles or sources to which you refer should be cited in-line with the authors’ names and the year of publication.\footnote{In-line citations are preferred over footnotes, and we favor APA citation format for both in-line citations and reference lists. Refer to the \href{https://owl.purdue.edu/owl/research_and_citation/apa_style/apa_formatting_and_style_guide/in_text_citations_the_basics.html}{Purdue Online Writing Lab}, or follow the above examples. Footnotes should use 8.5 point text with 1.26 line spacing.} The citation should be placed close in the text to the actual claim, not merely at the end of the paragraph. For example: students in the OMSCS program are older and more likely to be employed than students in the on-campus program \citep{joyner2017}. In the event of multiple authors, list them. For example: research finds sentiment analysis of the text of OMSCS reviews corresponds to student-assigned ratings of the course \citep{newman2018}. You may also cite multiple studies together. For example: several studies have found students in the online version of an undergraduate CS1 class performed equally with students in a traditional version (\cite{joyner2018a}; \cite{joyner2018b}). If you would like to refer to an author in text, you may also do so by including the year (in parentheses) after the author’s name in the text. If a publication has more than 4 authors, you may list the first author followed by ‘et al.’ For example: \citeinl{joyner2016} claim that a round of peer review prior to grading may improve graders’ efficiency and the quality of feedback given. This applies to parenthetical citations as well, e.g. \citep{joyner2016}.

\subsection{Reference lists}
References should be placed at the end of the paper in a dedicated section. Reference lists should be numbered and organized alphabetically by first author’s last name. If multiple papers have the same author(s) and year, you may append a letter to the end of the year to allow differentiated in-line text (e.g. Joyner, 2018a and Joyner, 2018b in the section above). If multiple papers have the same author(s), list them in chronological order starting with the older paper. Only works that are cited in-line should be included in the reference list. The reference list does not count against the length requirements.

\section{References}
\printbibliography[heading=none]

\section{Appendices}
You may optionally move certain information to appendices at the end of your paper, after the reference list. If you have multiple appendices, you should create a section with a \emph{Heading 1} of "Appendices." Each appendix should begin with a descriptive \emph{Heading 2}; appendices can thus be referenced in the body text using their heading number and description, e.g. "Appendix 5.1: Survey responses." If you have only one appendix, you can label it with the word "Appendix" followed by a descriptive title, e.g., "Appendix: Survey responses."

These appendices do not count against the page limit, but they should not contain any information required to answer the question in full. The body text should be sufficient to answer the question, and the appendices should be included only for you to reference or to give additional context. If you decide to move content to an appendix, be sure to summarize the content and note it in relevant place in the body text, e.g., "The raw data can be viewed in \emph{Appendix 5.1: Survey responses}."

\end{document}
