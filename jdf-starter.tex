\documentclass[
	%a4paper, % Use A4 paper size
	letterpaper, % Use US letter paper size
]{jdf}

\addbibresource{references.bib}

\author{Omair Tariq Khan}
\email{okhan60@gatech.edu}
\title{Homework 1}

\begin{document}
%\lsstyle

\maketitle
\newpage


\subsection{Answer to Question 1 - Selecting and Using Apple Music}
I use Apple Music in a variety of contexts, and I've noticed that each environment presents its own set of challenges, which significantly affect how I interact with the app. Whether I’m walking, sitting at home, or exercising, the user experience with the app shifts depending on my surroundings. When I’m walking down the street or navigating through a crowded area, I face external distractions—traffic, pedestrians, and the need to stay aware of my surroundings. These factors make it difficult for me to focus on the app itself. Additionally, my movements while walking can make precise touch interactions less reliable. In this context, the app could simplify the experience to reduce cognitive load and physical precision. For example, larger, more accessible buttons for essential functions like play/pause, skip, or volume control would be helpful. It would also be great if Apple integrated voice commands more prominently, allowing me to control the Music app hands-free while staying focused on the environment around me, especially in the case of non-English titles. 

In contrast, when I’m sitting at home or in a quiet space, I have more time and mental space to engage with Music. In this scenario, I tend to explore new albums and create playlists. The interface can afford to be more complex, offering detailed album artwork, extended playlist options, and deeper interactions with the app’s features. I enjoy a more immersive experience here, where I can spend time browsing and managing my music collection without distractions. 

However, when I’m exercising—whether at the gym or outdoors—the context changes again. I’m often sweaty, wearing gloves, or moving rapidly, which makes it harder to interact with small touch targets. I may also have limited hand dexterity during a workout. In this context, Apple Music could benefit from a more fitness-friendly interface. Larger buttons for quick access to song skipping, volume adjustments, and play/pause functions would be ideal. Also integrating with fitness trackers could allow me to tailor the music experience based on my workout intensity, keeping me motivated and focused without needing to manually adjust the app.

To improve the experience even further, I think Apple Music could be designed to adjust dynamically depending on my context. For instance, when I’m walking or driving, the app could simplify the interface, highlight larger controls, and prioritize voice commands. As a user, I'm more prone to be connected to Bluetooth in my earphones or my Car or Apple CarPlay when I'm not home. To add to my point from above, Apple Music has forward, backward, and pause buttons feature on Google Maps but not Apple Maps, which does highlight its ineffectiveness. At home, the app could offer more detailed options for music exploration, such as more intricate playlist management and album browsing. During exercise, the app could switch to a streamlined mode focused on large, easy-to-tap buttons or even be fully voice-controlled to allow for hands-free operation.

In conclusion, using Apple Music in different contexts introduces unique constraints, from external distractions to physical limitations. If it detects me walking or driving automatically, that would be great. I feel like with Siri handling basic commands like opening and closing the app itself, there just needs to be a bit more intuition on the app automatically adjusting based on these different environments. This would provide a more seamless and enjoyable experience, ensuring that I can interact with my music in the best possible way, no matter where I am. 
\newpage

\subsection{Answer to Question 2}
The process of submitting a question and receiving an answer on Ed Discussion involves several steps that highlight how the system supports or challenges students in achieving their goals. These steps can be analyzed using the concepts of the gulf of execution and the gulf of evaluation, focusing on how the platform helps bridge these gaps.

\subsubsection {Bridging the Gulf of Execution}
The gulf of execution begins with the student forming a goal, such as asking for clarification on a concept or receiving help with an assignment. Ed Discussion supports this planning stage by providing clear entry points for creating a new post. The “Ask a Question” button is prominently displayed on the interface, and categories such as “Homework Help” or “General Questions” act as signifiers, guiding students on where to post. However, for first-time users or those unfamiliar with the platform, these categories might be overwhelming or unclear. Adding a tooltip or a brief walkthrough when students first use the platform could make this step more intuitive.

Once the goal is formed, students move to the specification stage, where they determine the actions needed to accomplish their objective. Ed Discussion simplifies this by presenting a form with labeled fields for the title, question content, and category. Drop-down menus and clear labels help students specify their questions effectively. Despite this, some students might struggle with understanding how much detail to include or which category fits best. This could be addressed by showing example questions or providing templates that guide students in framing their questions appropriately.

Finally, students execute their plan by filling out the form, selecting the category, and submitting their question. Ed Discussion provides immediate visual feedback by displaying the question on the discussion board, tagged as “Unanswered.” This reassures students that their question has been successfully posted. However, for those who are new or uncertain, a preview of how the question will appear or a confirmation pop-up summarizing their submission could alleviate anxiety and improve confidence in the process.

\subsubsection {Bridging the Gulf of Evaluation}
The gulf of evaluation begins once the question is submitted, and students need to understand the outcome of their actions. Ed Discussion bridges the perception stage effectively by immediately displaying the submitted question under the selected category and tagging it as “Unanswered.” This visual feedback ensures that students can see their question and verify its presence on the platform. However, if the interface were less responsive or the feedback were delayed, students might doubt whether their question was posted successfully.

In the interpretation stage, students must understand the status of their question and what happens next. Ed Discussion supports this by making the question visible to peers, TAs, and professors, who can then reply. The “Unanswered” tag signals that the question is pending a response, helping students interpret the situation. However, some students may be unsure of the expected response time or whether their question was clear. Adding features like an estimated response time or a confirmation that the question was received and is being reviewed could enhance this stage.

The comparison stage involves evaluating whether the goal—receiving an answer—has been achieved. When a TA, professor, or peer responds, Ed Discussion notifies the student and displays the response below the original question in a threaded format. This clear organization helps students understand how the response addresses their query. However, if students are not notified promptly or struggle to locate the response, the comparison stage may falter. To address this, the platform could implement personalized notifications, such as email alerts or in-app badges, and highlight unread responses for better visibility.

Enhancing the Feedback Cycle
To further enhance the feedback cycle, Ed Discussion could incorporate dynamic feedback mechanisms. For example, hover-over tooltips on the “Ask a Question” button could provide students with a preview of what to expect after submitting their question. Immediate confirmation messages summarizing the next steps would reassure students that their submission was successful. Additionally, varying feedback methods, such as visual indicators showing the number of views or likes and auditory cues for new responses, could keep students informed and engaged.

By addressing the gulfs of execution and evaluation through thoughtful design, Ed Discussion can ensure a smoother and more intuitive experience for students. These improvements would not only help students achieve their goals but also foster a more effective and interactive learning environment.

\subsubsection{Heading 1}
\emph{Heading 1} should be set in all caps. It should have 11 points of space added before and 8.5 points of space added after

\subsubsection{Headings 2\,–\,4}
Besides \emph{Heading 1}, which is set in caps, headings should always use sentence case (i.e., first word capitalized) rather than title case; after all, they are not titles. \emph{Heading 2} should be set in bold roman (upright), and \emph{Heading 3} should be set in bold italics. The use of headings beyond \emph{Heading 3} is discouraged.

\subsubsubsection{Heading 4}
\emph{Heading 4} is provided as a run-in sidehead. Like \emph{Heading 3}, it is set in bold italics, but it should be followed by an em dash and flow right into the text, as seen at the beginning of the current paragraph. It should be used more as a list style than a heading style, e.g. to set off a list of principles in a heuristic evaluation.

\subsection{Page layout}
JDF uses the US Letter paper size (8.5" x 11"). It has a top margin of 1", and bottom and side margins of 1.5". This yields a text block of 5.5" x 8.5", which is exactly \(\frac{1}{2}\) the size of the page, divided lengthwise.

The page number should be included in the bottom margin, 1" from the bottom of the page. No other elements should be placed in the margins.

\section{Presentational elements}
You are encouraged to use presentational elements liberally, as long as they add to the clarity of your submissions. They often require less space and fewer accompanying words to explain a given concept, and do a far better job of it.

\subsection{Figures}
Figures should always be centered on the page, although they may also take up the entire width and height of the text block. Figures should always be referenced in the text, and they should include a descriptive caption. Figures may also be equations, diagrams, or other kinds of content.

If your figure includes a white background (e.g. an interface design or graph), it may aid legibility to add a \(\frac{1}{4}\)-point black border.

\begin{jdffigure}
\includegraphics[height=6cm]{Figures/flowchart.png}%
\captionof{figure}{Make sure your flowcharts are more useful than this one. Source: \href{https://xkcd.com/1195/}{XKCD}.}\label{fig:flowchart}%
\end{jdffigure}

Figure captions should be placed beneath the corresponding figure, indented 1" on the left and right sides. The label for the figure, e.g. "Figure 1," should be set in bold italics followed by an em dash, and the entire caption should be 8.5 points with 14 points of line spacing. Again, Word and LaTeX will number these automatically using the \emph{Figure Caption} paragraph style, but Docs users will need to number these manually. If need be, you may have one figure caption corresponding to multiple consecutive figures and use either locational descriptors (e.g. "top left," "middle") or labels (e.g. "A", "B") to map parts of the caption to parts of the figure. Make sure that caption falls on the same page as the corresponding figure or table; you may rearrange text to make this work.

In Microsoft Word, you may need to either change the image’s text wrap settings to "Top and Bottom" or change the line spacing of the image to 1.0.

\subsection{Tables}
You have freedom to format tables in the way that works best for your data. Generally, text should be left-aligned and numbers should be right-aligned or aligned at the decimal – you can do this using a \href{https://practicaltypography.com/tabs-and-tab-stops.html}{custom tab stop}. The default table style (shown below) reduces the text size to be equal to the caption text.

Table captions should be formatted the same way as figure captions, but they should be placed above the table. The popular mnemonic for this is: figures at the foot, tables at the top. Like figures, tables should not exceed the margins and should be centered on the page.

\begin{jdftable}
\captionof{table}{Mathematical constants. Notice how the approximations align at the decimal.}\label{table:Example}
\small % Reduce font size
\begin{tabular}{@{} L{0.15\linewidth} c S L{0.52\linewidth}}
	\textbf{Name} & \textbf{Symbol} & \textbf{Approximation} & \textbf{Description} \\
	\toprule[0.5pt]
	Golden ratio & $\phi$ & 1.618 & Number such that the ratio of " to the number is equal to the ratio of its reciprocal to 1\\
	\midrule
	Euler's number & $e$ & 2.71828 & Exponential growth constant\\
	\midrule
	Archimedes' constant & $\pi$ & 3.14 & The ratio between circumference and diameter of a circle\\
	\midrule
	One hundred & A+ & 100.00 & The grade we hope you’ll all earn in this class\\
\end{tabular}
\end{jdftable}

\subsection{Additional elements}
There are additional elements you may want to include in your paper, such as in-line or block quotes, lists, and more. For other content types not covered here, you have flexibility in determining how it should be used in this format.

\subsubsection{Quotes}
If you would like to quote an outside source, you may do so with quotation marks followed by a citation. If a quote is fewer than three lines, you may write it in-line. It is acceptable to replace pronouns with their target in brackets for clarity. For example, "Heavy use of peer grading would compromise [the school’s] reputation" \citep{joyner2016}. If a quote exceeds three lines, you should set it as its own paragraph with 0.5" side margins, using the \emph{Blockquote} style.

\begin{quotation}
"Whether or not the grades generated by peers are reliably similar to grades generated by experts is only one factor worth considering, however. Student perception is also an important factor. A recent study indicated that reliance on peer grading is one of the top drivers of high MOOC dropout rates. This problem may be addressed by reintroducing some expert grading where possible." \citep{joyner2016}
\end{quotation}

\subsubsection{Lists}
Bulleted and numbered lists are indented 0.5" from the left margin, with the bullet or number hanging in the margin by 0.25" (the default format).

Bullet points:

\begin{itemize}
	\item First bullet point item
	\item Second bullet point item
\end{itemize}

Numbered list:

\begin{enumerate}
	\item First numbered item
	\item Second numbered item
\end{enumerate}

\section{Procedural elements}
\subsection{In-line citations}
Articles or sources to which you refer should be cited in-line with the authors’ names and the year of publication.\footnote{In-line citations are preferred over footnotes, and we favor APA citation format for both in-line citations and reference lists. Refer to the \href{https://owl.purdue.edu/owl/research_and_citation/apa_style/apa_formatting_and_style_guide/in_text_citations_the_basics.html}{Purdue Online Writing Lab}, or follow the above examples. Footnotes should use 8.5 point text with 1.26 line spacing.} The citation should be placed close in the text to the actual claim, not merely at the end of the paragraph. For example: students in the OMSCS program are older and more likely to be employed than students in the on-campus program \citep{joyner2017}. In the event of multiple authors, list them. For example: research finds sentiment analysis of the text of OMSCS reviews corresponds to student-assigned ratings of the course \citep{newman2018}. You may also cite multiple studies together. For example: several studies have found students in the online version of an undergraduate CS1 class performed equally with students in a traditional version (\cite{joyner2018a}; \cite{joyner2018b}). If you would like to refer to an author in text, you may also do so by including the year (in parentheses) after the author’s name in the text. If a publication has more than 4 authors, you may list the first author followed by ‘et al.’ For example: \citeinl{joyner2016} claim that a round of peer review prior to grading may improve graders’ efficiency and the quality of feedback given. This applies to parenthetical citations as well, e.g. \citep{joyner2016}.

\subsection{Reference lists}
References should be placed at the end of the paper in a dedicated section. Reference lists should be numbered and organized alphabetically by first author’s last name. If multiple papers have the same author(s) and year, you may append a letter to the end of the year to allow differentiated in-line text (e.g. Joyner, 2018a and Joyner, 2018b in the section above). If multiple papers have the same author(s), list them in chronological order starting with the older paper. Only works that are cited in-line should be included in the reference list. The reference list does not count against the length requirements.

\section{References}
\printbibliography[heading=none]

\section{Appendices}
You may optionally move certain information to appendices at the end of your paper, after the reference list. If you have multiple appendices, you should create a section with a \emph{Heading 1} of "Appendices." Each appendix should begin with a descriptive \emph{Heading 2}; appendices can thus be referenced in the body text using their heading number and description, e.g. "Appendix 5.1: Survey responses." If you have only one appendix, you can label it with the word "Appendix" followed by a descriptive title, e.g., "Appendix: Survey responses."

These appendices do not count against the page limit, but they should not contain any information required to answer the question in full. The body text should be sufficient to answer the question, and the appendices should be included only for you to reference or to give additional context. If you decide to move content to an appendix, be sure to summarize the content and note it in relevant place in the body text, e.g., "The raw data can be viewed in \emph{Appendix 5.1: Survey responses}."

\end{document}
